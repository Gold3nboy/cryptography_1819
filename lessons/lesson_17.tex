\chapter*{Lesson 17}

\section{Construction of a \textsc{cca}-secure \textsc{pke}}

This section exposes a construction of a \textsc{cca}-secure \textsc{pke} scheme, using hash-proof systems, membership-hardness, and the $n$-universality property.

Let $\Pi_1, \Pi_2$ be two distinct hash-proof systems for some \textsc{np} language $L$ and the range of $Prove_2$ supports labels ($L'=L||\{0, 1\}^\ell$).

Construct the \textsc{cca} scheme as follows: $\Pi := (\mathcal{KG}en, Enc, Dec)$
\begin{itemize}
    \item $ (\overbrace{(\omega_1,\omega_2)}^{pk}, \overbrace{(\tau_1,\tau_2)}^{sk}) \pickUAR \mathcal{KG}en(1^\lambda)\;,\;\;\;\;(\omega_1, \tau_1) \pickUAR Setup_1(1^\lambda), (\omega_2, \tau_2) \pickUAR Setup_2(1^\lambda)$
    \item Encryption routine: $Enc((\omega_1, \omega_2), m)$
    \begin{itemize}
        \item $ (y, x) \pickUAR Sample_1(1^\lambda)$
        \item $ \pi_1 \pickUAR Prove_1(\omega_1, y, x)$
        \item $ l := \pi_1 \cdot m$
        \item $ \pi_2 \pickUAR Prove_2(\omega_2, (y, l), x)$
        \item $ c := (c_1, c_2) = ((y, l), \pi_2)$
    \end{itemize}
    \item Decryption routine: $Dec((\tau_1, \tau_2), (c_1, c_2))$
    \begin{itemize}
        \item $\hat{\pi}_2 = Verify_2(\tau_2, c_1))$
        \item $\textsc{if } \hat{\pi}_2 \neq c_2 \textsc{ then output false}$
        \item Recall: $c_1 = (y, l)$
        \item $\hat{\pi}_1 = Verify_1(\tau_1, y)$
        \item $\textsc{output } l \cdot \hat{\pi}_1^{-1}$
    \end{itemize}
\end{itemize}

Correctness (assume $\hat{\pi}_i = \pi_i \forall i$):
\begin{align*}
    \hat{m} &= Dec(sk, Enc(pk, m)) \\
    &= Dec((\tau_1, \tau_2), Enc((\omega_1, \omega_2), m)) \\
    &= Dec((\tau_1, \tau_2), ((y, l), \pi_2)) \\
    &= l \cdot \hat{\pi}_1^{-1} \\
    &= \pi_1 \cdot m \cdot \hat{\pi}_1^{-1} \\
    &= m
\end{align*}

Some additional notes (may be incorrect):
\begin{itemize}
    \item The message space of the second prover is the range of the first prover.
    \item The message space of the first prover is a multiplicative group
    \item The message space of the second prover is a polylogarithmic language in ($\lambda$)
\end{itemize}

\begin{theorem} 
    Assuming $\pi_1$ is 1-universal, $\pi_2$ is 2-universal and $L$
    is Membership-hard, then the above scheme is \textsc{cca}-secure.
\end{theorem}

\begin{proof}
 Five different games will be defined:
\todo{"I diagrammi arriveranno con un ritardo che preciserò al più presto. Mi scuso per il disagio."

- Trenitalia style}

\end{proof}

%==================================================

\begin{lemma}
    \[
        \forall b, G_{0}(\lambda, b) \equiv G_{1}(\lambda, b)   
    \]
\end{lemma}

\begin{proof}
    This follows by the correctness of $\Pi_{1}$ and $\Pi_{2}$
    \begin{gather*}
        \Pi_{1}=\tau_{1}=Ver_{1}(\tau, y)\\
         \Pi_{2}= \tilde{\Pi}_{2}=Ver_{2}(\tau, y)
    \end{gather*}
    with probability 1 over the choice of $(\omega_{1}, \tau_{1}) \leftarrow\$
    Setup_{1}(1^{\lambda})$
    
    \begin{gather*}
        \Pi_{1} \leftarrow Prove_{1}(\omega, y, x), (\omega_{2},
        \tau_{1})\leftarrow\$ Setup_{2} (1^{\lambda})\\
        \Pi_{2} \leftarrow Prove(\omega_{2}, (y, l), x) \forall y \in L , l \in
        \P_{1}
    \end{gather*}
    
\end{proof}

\begin{lemma}
    \[
        \forall b, G_{1}(\lambda, b) \approx_{c} G_{2}(\lambda, b)   
    \]
\end{lemma}
\begin{proof}
    Straight forward reduction from membership hardness.
\end{proof}

\begin{lemma}
    \[
        \forall b, G_{2}(\lambda , b) \approx_{c} G_{3}(\lambda, b)
    \]
\end{lemma}
\todo{I'm not completely sure the next proof is complete}
\begin{proof}
    Recall that the difference between $G_{2}$ and $G_{3}$ is that 
    \[
        (g^{(i)}, l^{(i)}, \Pi_{2}^{(i)})\text{ such that }y^{(i)}\not\in L 
    \]
    are answered $\perp$ in $G_{3}$, instead in $G_{2}$
    \[
        \perp \text{ comes out as output} \Leftrightarrow
        \tilde{\Pi}^{(i)}_{2}=Ver(\tau, y^{(i)}\not=\Pi^{(i)}_{2})
    \]

    It's possible to distinguish \textbf{two cases} , looking at $c=(y, l,
    \Pi_{2})$ :
\begin{enumerate}
    \item if $(y^{(i)}, l^{(i)})=(y,l)$ and
        $\tilde{\Pi}^{(i)}_{2}=\Pi^{(i)}_{2}$, it outputs $\perp$ if  in the
        decryption scheme ($\Pi^{(i)}_{2} \not= \tilde{\Pi}^{(i)}_{2}$) it
        outputs $\perp$.\\

    \item \footnote{when decryption oracle doesn't output the challenge}
        otherwise $(y^{(i)}, l^{(i)})\not= (y,l)$ if $y^{(i)}\not\in L$ we want
        that $\Pi^{(i)}_{2}$ doesn't output exactly $Ver_{2}(\tau, (y^{(i)},
        l^{(i)}))$, but it should output $\perp$.
\end{enumerate}

\textbf{EVENT BAD:} If we look at the view of $\A$, the only information he
knows is 
\[
    (\omega_{2}, \tilde{\Pi}_{2}=Ver_{2}(\tau_{2}, y))
\]
for $y \not\in L$.\\
The value 
\[
Ver_{2}(\tau_{2}, (y^{(i)}, x^{(i)}))
\]
for $y^{(i)} \in L$ and
$(y^{(i)}, l^{(i)})\not=(y, l)$ is random.\\
So, 
\[
\P [ BAD ] =2^{-|\P_{2}|} 
\]

\end{proof}

\begin{lemma}
    \[
        \forall b , G_{3}(\lambda, b)\equiv G_{4}(\lambda, b)   
    \]
\end{lemma}

\begin{proof}
    If we look at the view of $\A$, the only information known about $\tau_{1}$
    is $\omega_{1}$, since the decryption oracle only computes for $y^{(i)} \in
    L$
    \[
        Ver_{1}(\tau, y^{(i)})=Prove_{1}(\omega_{1}, y^{(i)}, x^{(i)})
    \]

    By 1-universality, $\Pi=Ver_{1}(\tau ,y)$ for any $y \in L$ is random.
\end{proof}

\begin{lemma}
    \[
        G_{4}(\lambda, 0)\equiv G_{4}(\lambda, 1)   
    \]
\end{lemma}
\begin{proof}
    The challenge ciphertext is independent of $b$.
\end{proof}





%==================================================
\review{referencing something from another part of lesson 17}

\subsection{Instantiation of U-HPS (Universal Hash Proof System)}
$\exists r$ using a DDH language such that
$L_{DDH}=\{(c_1,c_2), \exists r | c_1=g_1^r, c_2=g_2^r\}$

Given a group $\G$ of order $q$ with $(g_1,g_2)$ as generators:
$$G_2=g^x , g_1=g , r=g $$ $$ (g,g^x,g^y,g^{xy})$$

% not sure if this is U-HPS or something random
Now we can define our U-HPS $\Pi:=$(Setup, Prove, Verify)

\begin{itemize}
    \item $Setup(1^\lambda)$: Pick $x_1,x_2 \leftarrow\$ \Z_q$ and define:\\ $\omega=h_1=g_1^{x_1}g_2^{x_2}$, $\tau=(x_1,x_2)$
    \item $Prove(\omega, \underbrace{(c_1,c_2)}_{y}, r)$ output $\Pi=\omega^2$
    \item $Verify(\tau, \underbrace{(c_1,c_2)}_{y})$ output $\widetilde{\Pi}=c_1^{x_1}c_2^{x_2}$ %pretty shitty tilde here, I need a bigger tilde to cover the full length of Pi -> \usepackage{stackengine} \usepackage{scalerel} added in macro
\end{itemize}

\textbf{Correctness:} $\Pi=\omega^2=(g_1^{x_1}g_2^{x_2})^r=g_1^{rx_1}g_2^{rx_2}=c_1^{x_1}c_2^{x_2}=\widetilde{\Pi}$

\begin{theorem}
    If $(c_1,c_2)\notin L_DDH$ the distribution $(\omega=h_1, \widetilde{\Pi}=Verify(\tau,(c_1,c_2)))$ is indistinguishable form a truly random string.
\end{theorem}

\begin{proof}
    Define a \textbf{MAP} $\mu(\overbrace{x_1,x_2}^{random})=(\omega,\Pi)=(g_1^{x_1},g_2^{x_2},c_1^{x_1},c_2^{x_2})$ it suffices to prove that $\mu$ is injective. This can easily be done with some constrains:
    $\mu'(x_1,x_2)=log_{g_1}(\mu(x_1,x_2))=(log_{g_1}(\omega),log_{g_1}(\Pi)$\\
    For $r_1\neq r_2$ then $c_1=g_1^{r_1},c_2=g_2^{r_2}=g^{\alpha r_2}$. For $\alpha=log_{g_2}g_1$ then $\Pi=c_1^{x_1}c_2^{x_2}=g_1^{r_1x_1}g_2^{r_1x_1+\alpha r_2x_2}$.\\
    \[\mu'(x_1,x_2)=
        \begin{pmatrix}
            z_1\\
            z_2
        \end{pmatrix}
        =
        \begin{pmatrix}
            1 & \alpha \\
            r_1 & r_2\alpha 
        \end{pmatrix}
        \cdot
        \begin{pmatrix}
            x_1\\
            x_2
        \end{pmatrix}
    \]

    \[ \text{Since Det}
    \begin{pmatrix}
        1 & \alpha \\
        r_1 & r_2\alpha 
    \end{pmatrix}=
    \alpha(r_2-r_1)\neq 0
    \text{ the map is injective.}
    \] 
\end{proof}
