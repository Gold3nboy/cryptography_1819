\chapter*{Lesson 17}

\section{Construction of a \textsc{cca}-secure \textsc{pke}}

This section exposes a construction of a \textsc{cca}-secure \textsc{pke} scheme, using hash-proof systems, membership-hardness, and the $n$-universality property.

Let $\Pi_1, \Pi_2$ be two distinct hash-proof systems for some \textsc{np} language $L$ and the range of $Prove_2$ supports labels ($L'=L||\{0, 1\}^\ell$).

Construct the \textsc{cca} scheme as follows: $\Pi := (\mathcal{KG}en, Enc, Dec)$
\begin{itemize}
    \item $ (\overbrace{(\omega_1,\omega_2)}^{pk}, \overbrace{(\tau_1,\tau_2)}^{sk}) \pickUAR \mathcal{KG}en(1^\lambda)\;,\;\;\;\;(\omega_1, \tau_1) \pickUAR Setup_1(1^\lambda), (\omega_2, \tau_2) \pickUAR Setup_2(1^\lambda)$
    \item Encryption routine: $Enc((\omega_1, \omega_2), m)$
    \begin{itemize}
        \item $ (y, x) \pickUAR Sample_1(1^\lambda)$
        \item $ \pi_1 \pickUAR Prove_1(\omega_1, y, x)$
        \item $ l := \pi_1 \cdot m$
        \item $ \pi_2 \pickUAR Prove_2(\omega_2, (y, l), x)$
        \item $ c := (c_1, c_2) = ((y, l), \pi_2)$
    \end{itemize}
    \item Decryption routine: $Dec((\tau_1, \tau_2), (c_1, c_2))$
    \begin{itemize}
        \item $\hat{\pi}_2 = Verify_2(\tau_2, c_1))$
        \item $\textsc{if } \hat{\pi}_2 \neq c_2 \textsc{ then output false}$
        \item Recall: $c_1 = (y, l)$
        \item $\hat{\pi}_1 = Verify_1(\tau_1, y)$
        \item $\textsc{output } l \cdot \hat{\pi}_1^{-1}$
    \end{itemize}
\end{itemize}

Correctness (assume $\hat{\pi}_i = \pi_i \forall i$):
\begin{align*}
    \hat{m} &= Dec(sk, Enc(pk, m)) \\
    &= Dec((\tau_1, \tau_2), Enc((\omega_1, \omega_2), m)) \\
    &= Dec((\tau_1, \tau_2), ((y, l), \pi_2)) \\
    &= l \cdot \hat{\pi}_1^{-1} \\
    &= \pi_1 \cdot m \cdot \hat{\pi}_1^{-1} \\
    &= m
\end{align*}

Some additional notes (may be incorrect):
\begin{itemize}
    \item The message space of the second prover is the range of the first prover.
    \item The message space of the first prover is a multiplicative group
    \item The message space of the second prover is a polylogarithmic language in ($\lambda$)
\end{itemize}

Theorem: Assuming $\pi_1$ is 1-universal, $\pi_2$ is 2-universal and $L$ is Membership-hard, then the above scheme is \textsc{cca}-secure.

Proof: Five different games will be defined, from $\textsc{Game}_{\Pi, A}^0$ up to $\textsc{Game}_{\Pi, A}^4$; the first game will be an analogous formalization of how the above \textsc{pke} scheme works. It shall be proven that, for either fixed b in $\{0, 1\}$:
\begin{equation*}
    \textsc{Game}_{\Pi, A}^0(\lambda, b) = \textsc{Game}_{\Pi, A}^1(\lambda, b) \approx_C \textsc{Game}_{\Pi, A}^2(\lambda, b) \approx_S \textsc{Game}_{\Pi, A}^3(\lambda, b) = \textsc{Game}_{\Pi, A}^4(\lambda, b)
\end{equation*}
and finally that $\textsc{Game}_{\Pi, A}^4(\lambda, 0) = \textsc{Game}_{\Pi, A}^4(\lambda, 1)$, therefore concluding that $\textsc{Game}_{\Pi, A}^0(\lambda, 0) \approx_C \textsc{Game}_{\Pi, A}^0(\lambda, 1)$, and proving this scheme is \textsc{cca}-secure.

\todo{Non sono per niente sicuro riguardo all'origine di $x$ ed $y$, né tantomeno dove sia definito il sampler per essi}

The games are defined as follows:

% Game 0
\begin{cryptogame}{pkecca0}{Original \textsc{cca} game}{pke-cca}

    % Challenger generates key pair and publishes the public key
    \receive{$((\omega_1, \omega_2), (\tau_1, \tau_2)) \pickUAR \mathcal{KG}en(1^\lambda)$}
    {$(\omega_1, \omega_2)$}
    {}

    % Adversary can query the challenger for polynomially-many decryptions
    \send{}{}{}
    \receive{}{}{}

    \postlevel

    % Adversary asks for a challenge
    \send{}{$m_0, m_1$}{}
    %Challenger replies with an encrypted message
    \receive{\shortstack[l]{
        $(x, y) \pickUAR Sample_1$ \\
        $\pi_1 \pickUAR Prove_1(\omega_1, y, x)$ \\
        $b \pickUAR \{0, 1\}$ \\
        $\pi_2 \pickUAR Prove_2(\omega_2, (y, (\pi_1 \cdot m_b)), x)$ }}
    {$c = ((y, (\pi_1 \cdot m_b)), \pi_2)$}
    {}

    \postlevel

    % Adversary can query the challenger for polynomially-many decryptions
    \send{}{}{}
    \receive{}{}{}

    \postlevel

    % Adversary guesses which message has been encrypted
    \send{}{$b'$}{\textsc{Output} 1 \textsc{iff} $b=b'$}

\end{cryptogame}

% Game 1: Use verifiers instead of provers
\begin{cryptogame}{pkecca1}{Use verifiers}{1}

    % Adversary can query the challenger for polynomially-many decryptions
    \send{}{}{}
    \receive{}{}{}

    \postlevel

    % Adversary asks for a challenge
    \send{}{$m_0, m_1$}{}
    %Challenger replies with an encrypted message
    \receive{\shortstack[l]{
        $(x, y) \pickUAR Sample_1$ \\
        $\pi_1 \pickUAR Verify_1(\tau_1, y)$ \\
        $b \pickUAR \{0, 1\}$ \\
        $\pi_2 \pickUAR Verify_2(\tau_2, (y, (\pi_1 \cdot m_b)))$ }}
    {$c = ((y, (\pi_1 \cdot m_b)), \pi_2)$}
    {}

    \postlevel

    % Adversary can query the challenger for polynomially-many decryptions
    \send{}{}{}
    \receive{}{}{}

    \postlevel

    % Adversary guesses which message has been encrypted
    \send{}{$b'$}{\textsc{Output} 1 \textsc{iff} $b=b'$}

\end{cryptogame}

% Game 2: Sample (y, x(?)) outside the language instead of inside
\begin{cryptogame}{pkecca2}{Sample statements outside the language}{2}

    % Adversary can query the challenger for polynomially-many decryptions
    \send{}{}{}
    \receive{}{}{}

    \postlevel

    % Adversary asks for a challenge
    \send{}{$m_0, m_1$}{}
    %Challenger replies with an encrypted message
    \receive{\shortstack[l]{
        $(x, y) \pickUAR \overline{Sample}_1$ \\
        $\pi_1 \pickUAR Verify_1(\tau_1, y)$ \\
        $b \pickUAR \{0, 1\}$ \\
        $\pi_2 \pickUAR Verify_2(\tau_2, (y, (\pi_1 \cdot m_b)))$ }}
    {$c = ((y, (\pi_1 \cdot m_b)), \pi_2)$}
    {}

    \postlevel

    % Adversary can query the challenger for polynomially-many decryptions
    \send{}{}{}
    \receive{}{}{}

    \postlevel

    % Adversary guesses which message has been encrypted
    \send{}{$b'$}{\textsc{Output} 1 \textsc{iff} $b=b'$}

\end{cryptogame}

\todo{ Non è stato chiaro sull'origine di x ed y, inoltre mi manca da scrivere le query di decifratura}

% Game 3: Decryption queries return false whenever y is not in the language
\begin{cryptogame}{pkecca3}{Modify decryption queries}{3}

    % Adversary can query the challenger for polynomially-many decryptions
    \send{}{}{}
    \receive{}{}{}

    \postlevel

    % Adversary asks for a challenge
    \send{}{$m_0, m_1$}{}
    %Challenger replies with an encrypted message
    \receive{\shortstack[l]{
        $(x, y) \pickUAR \overline{Sample}_1$ \\
        $\pi_1 \pickUAR Verify_1(\tau_1, y)$ \\
        $b \pickUAR \{0, 1\}$ \\
        $\pi_2 \pickUAR Verify_2(\tau_2, (y, (\pi_1 \cdot m_b)))$ }}
    {$c = ((y, (\pi_1 \cdot m_b)), \pi_2)$}
    {}

    \postlevel

    % Adversary can query the challenger for polynomially-many decryptions
    \send{}{}{}
    \receive{}{}{}

    \postlevel

    % Adversary guesses which message has been encrypted
    \send{}{$b'$}{\textsc{Output} 1 \textsc{iff} $b=b'$}

\end{cryptogame}

% Game 4: pi1 is chosen UAR
\begin{cryptogame}{pkecca4}{$\pi_1$ is chosen \textsc{uar}}{4}

    % Adversary can query the challenger for polynomially-many decryptions
    \send{}{}{}
    \receive{}{}{}

    \postlevel

    % Adversary asks for a challenge
    \send{}{$m_0, m_1$}{}
    %Challenger replies with an encrypted message
    \receive{ \shortstack[l]{
        $\pi_1 \pickUAR Im(Prove_1)$ \\
        $b \pickUAR \{0, 1\}$ \\
        $\pi_2 \pickUAR Prove_2(\omega_2, (y, (\pi_1 \cdot m_b)), x)$
    }}
    {$c = ((y, (\pi_1 \cdot m_b)), \pi_2)$}{}

    \postlevel

    % Adversary can query the challenger for polynomially-many decryptions
    \send{}{}{}
    \receive{}{}{}

    \postlevel

    % Adversary guesses which message has been encrypted
    \send{}{$b'$}{\textsc{Output} 1 \textsc{iff} $b=b'$}

\end{cryptogame}

%==================================================

\review{referencing something from another part of lesson 17}

\subsection{Instantiation of U-HPS (Universal Hash Proof System)}
$\exists r$ using a DDH language such that
$L_{DDH}=\{(c_1,c_2), \exists r | c_1=g_1^r, c_2=g_2^r\}$

Given a group $\G$ of order $q$ with $(g_1,g_2)$ as generators:
$$G_2=g^x , g_1=g , r=g $$ $$ (g,g^x,g^y,g^{xy})$$

% not sure if this is U-HPS or something random
Now we can define our U-HPS $\Pi:=$(Setup, Prove, Verify)

\begin{itemize}
    \item $Setup(1^\lambda)$: Pick $x_1,x_2 \leftarrow\$ \Z_q$ and define:\\ $\omega=h_1=g_1^{x_1}g_2^{x_2}$, $\tau=(x_1,x_2)$
    \item $Prove(\omega, \underbrace{(c_1,c_2)}_{y}, r)$ output $\Pi=\omega^2$
    \item $Verify(\tau, \underbrace{(c_1,c_2)}_{y})$ output $\widetilde{\Pi}=c_1^{x_1}c_2^{x_2}$ %pretty shitty tilde here, I need a bigger tilde to cover the full length of Pi -> \usepackage{stackengine} \usepackage{scalerel} added in macro
\end{itemize}

\textbf{Correctness:} $\Pi=\omega^2=(g_1^{x_1}g_2^{x_2})^r=g_1^{rx_1}g_2^{rx_2}=c_1^{x_1}c_2^{x_2}=\widetilde{\Pi}$

\begin{theorem}
    If $(c_1,c_2)\notin L_DDH$ the distribution $(\omega=h_1, \widetilde{\Pi}=Verify(\tau,(c_1,c_2)))$ is indistinguishable form a truly random string.
\end{theorem}

\begin{proof}
    Define a \textbf{MAP} $\mu(\overbrace{x_1,x_2}^{random})=(\omega,\Pi)=(g_1^{x_1},g_2^{x_2},c_1^{x_1},c_2^{x_2})$ it suffices to prove that $\mu$ is injective. This can easily be done with some constrains:
    $\mu'(x_1,x_2)=log_{g_1}(\mu(x_1,x_2))=(log_{g_1}(\omega),log_{g_1}(\Pi)$\\
    For $r_1\neq r_2$ then $c_1=g_1^{r_1},c_2=g_2^{r_2}=g^{\alpha r_2}$. For $\alpha=log_{g_2}g_1$ then $\Pi=c_1^{x_1}c_2^{x_2}=g_1^{r_1x_1}g_2^{r_1x_1+\alpha r_2x_2}$.\\
    \[\mu'(x_1,x_2)=
        \begin{pmatrix}
            z_1\\
            z_2
        \end{pmatrix}
        =
        \begin{pmatrix}
            1 & \alpha \\
            r_1 & r_2\alpha 
        \end{pmatrix}
        \cdot
        \begin{pmatrix}
            x_1\\
            x_2
        \end{pmatrix}
    \]

    \[ \text{Since Det}
    \begin{pmatrix}
        1 & \alpha \\
        r_1 & r_2\alpha 
    \end{pmatrix}=
    \alpha(r_2-r_1)\neq 0
    \text{ the map is injective.}
    \] 
\end{proof}