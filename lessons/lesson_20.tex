\mychapter{20}{Lesson 20} %181205
\section{Random Oracle Model (ROM)}
The Random Oracle Model treats a given hash function $\H$ as a truly random function.

Recall: a truly random function $\F$ is defined to have a specific evaluation behaviour. Between subsequent evaluations:

\begin{itemize}
    \item if the argument hasn't been submitted to the function beforehand, then a value is chosen \uar{} from the codomain, and assigned as the image of said argument in the function;
    \item otherwise, the function will return the image as assigned in the corresponding previous evaluation.
\end{itemize}

They do act, in fact, as truth tables\footnotemark.

\footnotetext{Such tables are also aptly called \emph{rainbow tables}.}

\section{Full domain hashing}

Let $(f, f^{-1}, \Gen)$ be a \tdp{} scheme over some domain $\X_{pk}$

Take \rsa:
\begin{itemize}
    \item $(m, pk, sk) \pickUAR \Gen\rsa(1^\lambda)$
    \item $f(pk, x) = x^{pk} \mod n$
    \item $f^{-1}(sk, y) = y^{sk} \mod n$
\end{itemize}

Build a similar asymmetric-authentication scheme as such:

\begin{itemize}
    \item $(m, pk, sk) \pickUAR \Gen\rsa(1^\lambda)$
    \item $Sign_{sk, H}(x): \sigma = f^{-1}(sk, H(m))$
    \item $Verify_{pk, H}: H(m) = f(pk, \sigma)$
\end{itemize}

Exercise: show \rsa-sign is not secure without $H$. (Hint: The scheme becomes malleable)

Theorem: If the above scheme (full-domain hash) uses a \tdp{} for $f, f^{-1}$, then it is (asymmetric)-\ufcma under the random oracle model.

Proof: Reduce to \tdp, program the random oracle.

\begin{cryptoredux}
    {fdhufcma}
    {}
    {tdp}
    {ufcma(a)}

    \receive{\shortstack[l]{
        $(pk, sk) \pickUAR \Gen(1^\lambda)$ \\
        $x \pickUAR \X_{pk}$ \\
        $y = f(pk, x)$
    }}{$pk, y$}{}

    \invoke{$*$}{$pk$}{}

    \cseqdelay
    \cseqbeginloop
    \return{}{$m$}{}
    \invoke{}{$\sigma$}{}
    \cseqendloop
    \cseqdelay

    \return{}{$(m^*, \sigma^*)$}{}

    \send{}{$\sigma^*$}{}
\end{cryptoredux}

Notes: this is a loose reduction

Some assumptions are made:

\begin{itemize}
    \item The adversary makes the same number of RO queries as the number of signing queries done by the distinguisher (without loss of generality)
    \item The RO query must be done \emph{before} the corresponding sign query, otherwise the advarsary cannot sign the messages, as specified by the scheme
\end{itemize}

The RO queries are actually an analogue of the definition of a random function, and it is the \emph{programming} step of the oracle itself; then if the signing queries do not correspond to any RO query, abort the game.

\section{ID Scheme}

\section{Honest Verifier Zero Knowledge (HVZK) and Special Soundness (SS)}

\subsection{Fiat-Shamir scheme}