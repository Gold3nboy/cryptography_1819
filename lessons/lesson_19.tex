\mychapter{19}{Lesson 19} %181130
\todo{Warining: e' venuto fuori un casino in questa lezione, ho cercato di riordinare le cose}
\section{Bilinear Map}
Let's define a \textbf{Bilinear Group} as $(\G, \G_t, q, g, \hat{e})\leftarrow BilGen(1^{\lambda})$ where:
\begin{itemize}
    \item $\G, \G_t$ are prime order groups (order $q$).
    \item $g$ is a random generator of $\G$.
    \item $(\G, \cdot)$ is a multiplicative group and $\G_t$ is called \underline{target} group.
    \item $\hat{e}$ is a (bilinear) MAP: $\G \times \G \rightarrow \G_t$ efficiently computable. Defined as follows:
    
    \centering{$\rightarrow$ Take a generator $g$ for $\G$ and an element $h$ in $\G$.\\
    $\forall g,h \in \G, a, b \in \Z_q \hat{e}(g^a,h^b)=\hat{e}(g,h)^{ab}=\hat{e}(g^{ab},h)$ \\and $\hat{e}(g,g)\neq 1$ (Non degenerative)}
\end{itemize}
"I can move the exponents"

Venturi said something here related to Weil pairing over an elliptic curve. I found \href{https://www.math.auckland.ac.nz/~sgal018/crypto-book/ch26.pdf}{this}. Interesting but not useful.
 
\noindent\textbf{Assumption:} CDH is HARD in $\G$.\\
\textbf{Observation:} DDH is EASY in $\G$!
\begin{proof}
    $(g,g^a,g^b,g^c)\approx_c (g,g^a,g^b,g^{ab}) \implies \hat{e}(g^a,g^b)=_?\hat{e}(g,g^c)$ now just move the exponents and I can transform the first element in $\hat{e}(g,g^{ab})$ and check if it is equal to $\hat{e}(g,g^c)$.
\end{proof}

Now $KGen(1^{\lambda})$ will:
\begin{itemize}
    \item Generate some params $ =(\G,\G_t,g,q,\hat{e})\leftarrow BilGen(1^{\lambda})$ 
    \item Pick $a \pickUAR \Z_q$ then $g_1=g^a$
    \item Pick $g_2=g^b$ and $g_2,u_0,u_1,...,u_k \pickUAR \G$.
    \item Then output: 
    \begin{itemize}
        \item $P_k=(params, g_1,g_2,u_0,...,u_k)$
        \item $S_k=g_2^a=g^{ab}$
    \end{itemize}
\end{itemize}

Sign($S_k$, m):
\begin{itemize}
    \item Divide the message $m$ of length $k$ in bits and not it as follows: $m=(m[1],...,m[k])$
    \item Now define $\alpha(m)=u_0\prod_{i=1}^k u_i^{m[i]}$
    \item Pick $r\leftarrow \Z_q$ and output the signature $\sigma =(\underbrace{g_2^a}_{S_k} \cdot \alpha(m)^r,g^r)=(\sigma_1,\sigma_2)$
\end{itemize}

Vrf($P_k$, m, ($\sigma_1,\sigma_2$))
\begin{itemize}
    \item Check $\hat{e}(g,\sigma_1)=\hat{e}(\sigma_2,\alpha(m))=\hat{e}(g_1,g_2)$
\end{itemize}

\textbf{Correctness:} \textit{"Just move the exponents"} 

$\hat{e}(g,\sigma_1)=\overbrace{\hat{e}(g,g_2^a\cdot \alpha(m)^a)}^{\text{I can separate the second part for bilinearity}}=\hat{e}(g,g_2^a)\cdot \hat{e}(g,\alpha(m)^r)=\hat{e}(\underbrace{g^a}_{g_1},g_2)\cdot \hat{e}(\underbrace{g^r}_{\sigma_2},\alpha(m))=\hat{e}(g_1,g_2)\cdot\hat{e}(\sigma_2,\alpha(m))$\\
We can say that we are moving the exponents from the "private domain" to the "public domain".

\section{Waters signatures}
\begin{theorem}
    The Waters' signature scheme is \ufcma
\end{theorem}

\begin{proof}
    \begin{quote}
        The trick is to ``program the 'u's '' (Venturi)
    \end{quote}

    \todo{Sequence is incomplete/incorrect, have to study it more...}

    \begin{figure}[ht]
        \centering
        \sdinit{}
        \begin{tikzpicture}
            
            \sdbegin{}
            \newinst{A}{A}
            \newinst[2.5]{B}{C$^\textsc{wds}$}
            \newinst[2.5]{C}{C$^\cdh$}
           
            \postlevel

            % Instantiate game
            \mess{C}{$(\text{``params''}, g^a, g^b)$}{B}
            \node[anchor=west] at (mess from) {\shortstack[l]{
                ``params''$ = $ \\
                $ \quad =(\G, \G_t, g, q, \hat{e})$ \\
                $ \quad \pickUAR BilGen(1^\lambda)$ \\
                $a, b \pickUAR \mathbb{Z}_q$}};
           
           % Send pk to adversary
           \postlevel
           \mess{B}{$(\text{``params''}, g^a, g^b, u)$}{A}
           \node[anchor=west] at (mess from) {Construct $u$ string};

           % Adversary can query for signatures
           \postlevel
           \mess{A}{$m$}{B}
           \node[anchor=east] at (mess from) {$m \in M$};
           \mess{B}{$\sigma$}{A};
           \node[anchor=west] at (mess from) {$\sigma = \textit{Sign}()$};

           % Adversary forges
           \postlevel
           \mess{A}{$(m^*, \sigma^*)$}{B}
           \node[anchor=east] at (mess from) {$m^* \notin M$};
           
           \sdend{}
        \end{tikzpicture}
        \caption{Reducing Waters' scheme to \cdh}
        \label{fig:reduxwaters}
    \end{figure}

    \todo{The following explanation is roundabout, will rectify later}

    The following describes how the Waters' challenger constructs the $u$ string. The main idea is, given $k$ as the message length, to choose each single bit of $u$ from 1 up to $k$ such that:

    \begin{equation*}
        \alpha(m) = g_2^{\beta(m)}g^{\gamma(m)},\quad \beta(m) = x_0 + \sum_{i=1}^{k}m_ix_i,\quad \gamma(m) = y_0 + \sum_{i=1}^{k}m_iy_i
    \end{equation*}

    where $x_0 \pickUAR \{-kl, \dots, 0\}, x_{1\text{---}k} \pickUAR \{0, \dots, l\}, y_{0\text{---}k} \pickUAR \mathbb{Z}_q$

    In particular: $l = 2q_s$, where $q_s$ is the number of sign queries made by the adversary.

    Therefore, let $u_i = g_2^{x_i}g^{y_i} \quad \forall i \in [0, k]$. Then:

    \begin{equation*}
        \alpha(m) = g_2^{x_0}g^{y_0} \prod_{i=1}^k (g_2^{x_i}g^{y_i})^{m_i} = g_2^{x_0+\sum_{i=1}^k m_ix_i}g^{y_0+\sum_{i=1}^k m_iy_i} = g_2^{\beta(m)}g^{\gamma(m)}
    \end{equation*}

    \todo{Partizioni, doppi if.... qui non ci ho capito 'na mazza}

    \textbf{Step 1}: $\sigma = (\sigma_1, \sigma_2) = (g_2^a \alpha(m)^{\bar{r}}, g^{\bar{r}})$, for $\bar{r} \pickUAR \mathbb{Z}_q \quad \bar{r} = r - a\beta^{-1}$

    \begin{align*}
        \sigma_1 &= g_2^a\alpha(m)^{\bar{r}} \\
        &= g_2^a\alpha(m)^{r - a\beta^{-1}} \\
        &= g_2^a (g_2^{\beta(m)} g^{\gamma(m)})^{r - a\beta^{-1}} \\
        &= g_2^a g_2^{\beta(m)r-a} g^{\gamma(m)r - \gamma(m)a\beta^{-1}} \\
        &= g_2^{\beta(m)r} g^{\gamma(m)r} g^{-\gamma(m)\beta^{-1}}
    \end{align*}

\end{proof}

%https://eprint.iacr.org/2011/703.pdf