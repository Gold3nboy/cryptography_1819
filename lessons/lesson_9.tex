\mychapter{9}{Lesson 9}

\section{Message Authentication and UFCMA-security}

First, remember the $Tag()$ function and how a MAC works.

Now $Tag()$ is defined using a key $k$, and we call it $Tag_{k}()$.

In particular we are looking for a cool property of a message authentication protocol, called \textbf{ \textit{universal unforgeability against chosen-message attacks}}, which prevents the attacker from generating a valid couple $(m^{*}, \phi^{*})$ after some queries containing messages and receiving the related tags.

This property is defined through a game called 
\[
    Game_{\Pi, \A}^{ufcma}(\lambda)
\]
and played in the following manner:

\begin{figure}[h!]
   \centering
   \sdinit{}
   \begin{tikzpicture}
      % Define symbols and names for the parties
      \sdbegin{}
      \newinst{A}{$ \A $}
      \newinst[5]{B}{$ C_{k \leftarrow\$ \K} $} % Increase "5" to widen
      
      \postlevel
      \mess{A}{$m_{i}$}{B}
      \node[anchor=east] at (mess from) {  };
      \postlevel
      \mess{B}{$\phi_{i}$}{A}
      \node[anchor=west] at (mess from) {$\phi_{i}=Tag_{k}(m_{i})$};
    \draw [->] (1.2,-3.3) to[out=240,in=110] (1.2,-1.7);
      
   \postlevel 
   \mess{A}{$(m^{*}, \phi^{*})$}{B}
   \node[anchor=west] at (mess to) {Output 1 IFF $Tag_{k}(m^{*})=\phi^{*}$ };

      \sdend{}
   \end{tikzpicture}
   \caption{$Game_{\Pi, \A}^{ufcma}(\lambda)$}
\end{figure}

with $m^{*}$ which must be a fresh new message never used before and with the Adversary $\A$ which doesn't know the key $k$ used in the tag.

\begin{definition}
    $\Pi$ is \textbf{ \textit{ufcma-secure} } if $ \forall.PPT.\A$ 
    \[
        \P [ Game_{\Pi, \A}^{ufcma}(\lambda)=1 ] \in \negl{\lambda}  
    \]
\end{definition}

Now consider the following theorem:
\begin{theorem}
    Let $\F=\{F_{k}:\{0,1\}^{n} \to \{0,1\}^{l}\}_{k \in \{0,1\}^{\lambda}}$ 
    be a PRF family.\\
    Then $\Pi$ which uses $Tag_{k}()=F_{k}()$ is a UFCMA-secure
    MAC with n-bit domain.
\end{theorem}
We show that the scheme which uses a PRF is indistinguishable from a scheme
which uses a random function, and that a MAC scheme which uses a random function
is breakable by an efficient attacker in negligible time.
\begin{proof}
Consider the two following hybrids:

\begin{figure}[h!]
   \centering
   \sdinit{}
   \begin{tikzpicture}[scale=0.6]
      % Define symbols and names for the parties
      \sdbegin{}
      \newinst{A}{$ \A $}
      \newinst[5]{B}{$ C_{k \leftarrow\$ \K} $} % Increase "5" to widen
      
      \postlevel
      \mess{A}{$m_{i}$}{B}
      \node[anchor=east] at (mess from) {  };
      \postlevel
      \mess{B}{$\phi_{i}$}{A}
      \node[anchor=west] at (mess from) {$\phi_{i}=F(k,m_{i})$};
    \draw [->] (1.2,-3.3) to[out=240,in=110] (1.2,-1.7);
      
   \postlevel 
   \mess{A}{$(m^{*}, \phi^{*})$}{B}
   \node[anchor=west] at (mess to) {Output 1 IFF $F(k,m^{*})=\phi^{*}$ };

      \sdend{}
   \end{tikzpicture}
   \caption{$Game_{\F, \A}^{ufcma}(\lambda)$}
\end{figure}

\begin{figure}[h!]
   \centering
   \sdinit{}
   \begin{tikzpicture}[scale=0.6]
      % Define symbols and names for the parties
      \sdbegin{}
      \newinst{A}{$ \A $}
      \newinst[5]{B}{$ C_{R \leftarrow\$ \R(\lambda, n, l)} $} % Increase "5" to widen
      
      \postlevel
      \mess{A}{$m_{i}$}{B}
      \node[anchor=east] at (mess from) {  };
      \postlevel
      \mess{B}{$\phi_{i}$}{A}
      \node[anchor=west] at (mess from) {$\phi_{i}=R(m_{i})$};
    \draw [->] (1.2,-3.3) to[out=240,in=110] (1.2,-1.7);
      
   \postlevel 
   \mess{A}{$(m^{*}, \phi^{*})$}{B}
   \node[anchor=west] at (mess to) {Output 1 IFF $R(m^{*})=\phi^{*}$ };

      \sdend{}
   \end{tikzpicture}
   \caption{$\H\Y\B^{ufcma}(\lambda)$}
\end{figure}

\begin{lemma}
 $Game_{\F, \A}^{ufcma}(\lambda) \approx_{c} \H\Y\B_{1}^{ufcma}(\lambda)$  
\end{lemma}
Assume that these two are distinguishable by D.So we could have D' which is
capable , with the following game, of distinguish a pseudo-random function from
a true random function:
\newpage


\begin{figure}[h!]
   \centering
   \sdinit{}
   \begin{tikzpicture}[scale=0.5]
      % Define symbols and names for the parties
      \sdbegin{}
      \newinst{A}{$ \D $}
      \newinst[5]{B}{$ \D_{'} $} % Increase "5" to widen
      \newinst[4]{C}{$ \C^{prf}_{k \leftarrow\$ \{0,1\}^{\lambda} / R
      \leftarrow\$ \R(\lambda, n, l)} $}

      \postlevel
      \mess{A}{$m_{i} \in \M$}{B}
      \node[anchor=west] at (mess to) {  };
      \postlevel
      \mess{B}{$m_{i}$}{C}
      \node[anchor=west] at (mess to) {  };

      \postlevel
      \mess{C}{$\phi_{i}$}{B}
      \node[anchor=west] at (mess from) {\shortstack[l] {
                  $\phi_{i}= F_{k}(m_{i})$
                  \\
                  $\phi_{i}=R(m_{i})$
          } };
      \postlevel:
      \mess{B}{$\phi_{i} $}{A}
      \node[anchor=west] at (mess to) {  };
    \draw [->] (1.2,-5.7) to[out=240,in=110] (1.2,-1.6);
    \postlevel
    \mess{A}{$m^{*}, \phi^{*}$}{B}
      \node[anchor=west] at (mess to) {  };

      \postlevel((
    \mess{B}{$m^{*}$}{C}
      \node[anchor=west] at (mess to) {  };
    
    \postlevel
    \mess{C}{$\phi'$}{B}
      \node[anchor=west] at (mess to) {  };
      \postlevel
      \mess{B}{$ \phi^{*}=^{?}\phi'$}{A}
      \node[anchor=west] at (mess from) {\shortstack[l]{
      		$    $ 
            \\
            $    $ }};
    \postlevel
    \mess{A}{$b'$}{B}
      \node[anchor=west] at (mess to) {  };
      
    \postlevel
    \mess{B}{$b'$}{C}
      \node[anchor=west] at (mess to) {  };
      
%      \postlevel
%      \mess{A}{$b' \in {0,1}$}{B}
%      \node[anchor=west] at (mess to) {  };
      \sdend{}
      \sdend{}
   \end{tikzpicture}
  % \caption{}
  % \label{fig:}
\end{figure}

After D' receive $\phi'$ from C, he has to use the distinguisher D to distinguish the PRF
function from the random one.\\
So he says to D if $\phi^{*}=\phi'$ or not, and D now is capable of
understanding which game he is playing.

\begin{lemma}
    For all efficient adversaries $ \P [ \H\Y\B_{1}(\lambda)=1 ] \leq 2^{-l}  $
\end{lemma}
This is true because attacker has to predict the output $R(m^{*})$ on a fresh
input $m^{*}$ and to send to the Challenger this couple to win the game.\\
This can happen, at most, with probability $\frac{1}{2^{l}}$.
\end{proof}

This was for fixed length messages, since we can encrypt messages long $n$.\\
In the next subsection, we will see how to create a UFCMA-secure MAC for
variable length messages.

\section{Domain extension}
Assume $m=(m_{1}, ..., m_{t}) \in (\{0,1\}^{n})^{t}$for some $t \geq 1$. How can we tag m given just $Tag_{k}:\{0,1\}^{n} \to \{0,1\}^{l} $?

We can try various attempts:
\begin{itemize}
    
    
    
    \item \label{xorall}XOR all blocks: $\phi=Tag_{k}(\bigxor_{i=1}^{t}m_{i})$.\\
        This is not secure cause given $(m, \phi)$ it is possible to find
       $m^{*}\not= m$ s.t.
       \[
\bigxor m_{i} = \bigxor m_{i}^{*}
        \]
        and output this the couple $(m^{*}, \phi)$ to win.

    


    \item Let $\phi_{i}=Tag_{k}(m_{i})$ and the final message of the challenge
        has this form : $(m, \phi=(\phi_{1}, ..., \phi_{t}))$.\\
        This is not secure. Given $m=(m_{1}, ..., m_{t})$, this message has an
        unique $\phi=(\phi_{1}, ..., \phi_{t})$, and if I swap $m_{1}$ and
        $m_{t}$ I obtain a fresh new message, with a fresh new tag
        $\phi'=(\phi_{t}, \phi_{2}, ..., \phi_{t-1}, \phi_{1})$.
        Using this new couple, the game is won.



    \item Try with $\phi_{i}=Tag_{k}(i||m_{i})$, authenticating the position of
        the block .\\
        But this is not secure, and can be showed in \textbf{just 2 queries}.
        (I solved this in class during the break, with $t+1$ queries: $t$ for
        retrieving the partial tag of the submessage+position , and the last
        query to merge all the obtained results in a fresh new message. But this
        solution can be improved.)
        (UPDATE: I send  $m=m_{1_{1}}, ...m_{1_{t}}$ and I save the
        corresponding $\phi$. Then I send $m'=m_{2_{1}}, ..., m_{2_{t}}$ and I
        save the $\phi'$. Now I forge the new fresh prince of Bel Air
        $m^{*}=m_{1}, m_{2}, m_{2}, ...,m_{2}$
        and I can forge also a valid $\phi^{*}$ because I have all the
        signed parts of this new tag. )
        

\end{itemize}

Now I feel cool in Los Angeles and I want to explore a new \textbf{IDEA}:\\
the design of a shrinking functions family 
\[
   \H= \{ h_{s}:\{0,1\}^{nt} \to \{0,1\}^{n} \}_{s \in \{0,1\}^{\lambda}}
\]
which can be used to shrink variable length messages and then apply a PRF on
them.

This idea is cool, and I consider the induced family
\[
    \F(\H)=\{F_{k}(h_{s}(.))\}
\]
\begin{question}
Which are the properties of this family?
\end{question}

The main problem are \textit{collisions} , since for each $m \in \{0,1\}^{nt}$
it should be hard to find $m' \not= m$ such that $h_{s}(m)=h_{s}(m')$.\\
But we know that collisions exist, because we are trying to create a function 
\[
    Tag_{k, s}(m)=F_{k}(h_{s}(m))
\]
which maps elements in $\{0,1\}^{nt} $ to the elements in $\{0,1\}^{t}$, and
since the second set is smaller, for the pidgeonhole principle, there will be
 elements of $\{0,1\}^{t}$ which will be reached by more than one element of
$\{0,1\}^{nt}$.

To overocome this problem, we can consider 2 ways:
\begin{itemize}
    \item assume collisions are hard to find given $s \in \{0,1\}^{\lambda}$
        publicly, and we have a \textit{collision resistant hashing};
    \item let $s$ be secret, and assume collisions are hard to find because it
        is hard to know how $h_{s}$ works.
\end{itemize}
\begin{definition}
    $\H$ is called \textbf{ \textit{universal} } family if 
    \[
        \forall x, x' \in \{0,1\}^{nt} \text{ such that } x \not= x'
    \]
    \[
        \P_{s \leftarrow\$ \{0,1\}^{\lambda}} [ h_{s}(x)=h_{s}(x') ] \leq \epsilon  
    \]
\end{definition}

For $\epsilon=2^{-n}$ we call it \textbf{perfectly universal}.

For $\epsilon \in \negl{\lambda} $ we call it \textbf{almost universal} (or \textbf{AU} ).

\begin{exercise}
    Show that any pairwise independent hash function is perfectly
    universal.(should I use $Col$ for solving this? What is the difference and
    when I should use $Col$ instead of one-shot-probability?) \textbf{ASK FOR
    SOLVING PROPERLY} (Thoughts: when I ask \textit{what's the probability that
, chosen 2 distinct x-es, their hashes are the same \textbf{on a certain value}?
}, maybe I have to use one-shot, because one-shot refers to the prob. that the
two inputs collide on a specific value, even if not specified.\\
Instead, if I consider \textit{what's the prob. that , chosen 2 distinct x-es,
their hashes are the same?}, maybe I have to calculate all the possible
collisions, because I want to know if the 2 inputs can collide in general. )
\end{exercise}

\begin{theorem}
    Assuming $\F$ is a PRF with $n-$bit domain and $\H$ is AU, then $\F'=\F(\H)$
    is a PRF  (and , if used in a MAC as tag function, makes it UFCMA) on
    $nt-$bit domain (for $t \geq 1$).
\end{theorem}
We want to show that $\F'=\F(\H)$ is a PRF , so we want to show that 
\[
    Real_{\F, \A}(\lambda) \approx_{c} Rand_{\R', \A}(\lambda)
\]
 \begin{proof}
     Consider these 3 experiments/games:

\begin{figure}[h!]
   \centering
   \sdinit{}
   \begin{tikzpicture}[scale=0.5]
      % Define symbols and names for the parties
      \sdbegin{}
      \newinst{A}{$ \A $}
      \newinst[5]{B}{$ C_{k \leftarrow\$ \{0,1\}^{\lambda}, s \leftarrow\$
      \{0,1\}^{\lambda} }$}
      
      \postlevel
      \mess{A}{$x$}{B}
      \node[anchor=east] at (mess from) {  };
      \postlevel
      \mess{B}{$y$}{A}
      \node[anchor=west] at (mess from) {$y=F_{k}(h_{s}(x))$};

    \draw [->] (1.2,-3.3) to[out=240,in=110] (1.2,-1.7);
      \sdend{}
   \end{tikzpicture}
   \caption{$Real_{\F, \A}(\lambda)$}
\end{figure}

\begin{figure}[h!]
   \centering
   \sdinit{}
   \begin{tikzpicture}[scale=0.5]
      % Define symbols and names for the parties
      \sdbegin{}
      \newinst{A}{$ \A $}
      \newinst[5]{B}{$ C_{R \leftarrow\$ \R(\lambda, n, l), s \leftarrow\$
      \{0,1\}^{\lambda} }$}
      
      \postlevel
      \mess{A}{$x$}{B}
      \node[anchor=east] at (mess from) {  };
      \postlevel
      \mess{B}{$y$}{A}
      \node[anchor=west] at (mess from) {$y=R(h_{s}(x))$};

    \draw [->] (1.2,-3.3) to[out=240,in=110] (1.2,-1.7);
      \sdend{}
   \end{tikzpicture}
   \caption{$\H\Y\B_{\R, \A}(\lambda)$}
\end{figure}

\begin{figure}[h!]
   \centering
   \sdinit{}
   \begin{tikzpicture}[scale=0.5]
      % Define symbols and names for the parties
      \sdbegin{}
      \newinst{A}{$ \A $}
      \newinst[5]{B}{$ C_{R'\leftarrow\$ \R(\lambda, n, l)}$}
      
      \postlevel
      \mess{A}{$x$}{B}
      \node[anchor=east] at (mess from) {  };
      \postlevel
      \mess{B}{$y$}{A}
      \node[anchor=west] at (mess from) {$y=R(x)$};

    \draw [->] (1.2,-3.3) to[out=240,in=110] (1.2,-1.7);
      \sdend{}
   \end{tikzpicture}
   \caption{$Rand_{\R', \A}(\lambda)$}
\end{figure}

\begin{lemma}
    
    $Real \approx_{c} \H\Y\B$

\end{lemma}
\begin{exercise}
    Prove it!
\end{exercise}

\begin{lemma}
    \[
       \H\Y\B \approx_{c} Rand 
    \]
\end{lemma}
These 2 experiments are very similar but for the encryption step.\\
If I send 2 consecutives queries to both the experiments, while $Rand$ will
give me 
\begin{itemize}
    \item the same $y$ with the same $x$ or 
    \item the same $y$ with different $x_{1}\not=x_{2}$ but with very low probability, 
\end{itemize}
$\H\Y\B$ could return the same $y$:
\begin{itemize}
    \item with the same $x$ input or
    \item with different $x_{i}\not=x_{j} \Rightarrow h_{s}(x_{i}) 
        \not= h_{s}(x_{j})$ but with very low
        probability (because $R$ is a random function), or
    \item with different $x_{i}\not= x_{j}$ but with the same
        $h_{s}(x_{i})=h_{s}(x_{j})$,
\end{itemize}

We want to show that the last item doesn't happen too often, and that these 2
experiments are distinct for a negligible factor.

\begin{proof}
    Let \textbf{BAD}  be the event that
    \begin{equation*}
        \exists i, j \in [q] \text{ with } i\not= j \text{ s.t. } h_{s}(x_{i})=h_{s}(x_{j})
    \end{equation*}

    As logn as \textbf{BAD} doesn't happen, the function $R$ is run as a
    sequence of distinct points $R(h_{s}(x_{1})), ..., R(h_{s}(x_{q}))$.

    So in this case the distribution if the 2 games is identical, and it
    suffices to show that $ \P [ BAD ] \in \negl{\lambda}  $.

    The event \textbf{BAD}, which happens during a game made of $q$ queries and
    $q$ replies ,  is the same as collecting all the $q$ queries, chosing the
    seed and then looking for collisions. If interpreted in this way, 
    \begin{gather*}
        \P [ BAD ] = \P_{s} [ \exists x_{i}\not=x_{j}, h_{s}(x_{i})=
        h_{s}(x_{j})] \leq \\
        \leq \sum_{i,j} \underbrace{\P [ h_{s}(x_{i})=h_{s}(x_{j})
        ]}_{h_{s}\text{ is AU by definition }} \leq \\
        \leq {q \choose 2} \negl{\lambda} \in \negl{\lambda}    
    \end{gather*}

\end{proof}

So now we have $Real \approx_{c} \H\Y\B \approx_{c} Rand$
 \end{proof}
    
 Now we want to show a possible shrinking/hash function which can be used in a
 UFCMA-secure MAC. In order to be used in a UFCMA-secure MAC as the input of a
 PRF, this function must be part of an \textbf{\textit{almost universal}} family.

 \subsection{$\H$ family using Galois Fields}
 \begin{construction}
     Take $ \mathbb{F}=GF(2^{n})$, a \textit{Galois field} of $2^{n}$
     elements.\\
     Let $m=(m_{1}, ..., m_{t}) \in \mathbb{F}^{t} $ and $s=(s_{1}, ...,
     s_{t}) \in \mathbb{F}^{t} $. We state that 
     \[
         h_{s}(m)= \sum_{i=1}^{t}s_{i}m_{i}=<s,m>=q_{m}(s)
        \]
 \end{construction}

 A generic \textit{galois field} has very interesting properties, like the
 following :
 \begin{itemize}
     \item addition of two elements is like applying the XOR operation on their
         binary forms;
     \item multiplication of two elements is like the product $mod 2^{n}$
 \end{itemize}.

 Now, since this family must be almost universal to be used as part of
 UFCMA-secure MAC protocol, collisions must happen a negligible amount of
 time.\\
 Suppose we have a collision with two different messages:
\[
     \sum_{i=1}^{t} m_{i}s_{i}=\sum_{i=1}^{t} m'_{i}s_{i}
\]
 Let $\delta_{i}=m_{i}-m'_{i}$, assuming  without loss of generality that
 $\delta \not= 0$.\\
 Now we have that, when a collision happens, 
 \begin{gather*}
     \sum_{i=1}^{t} m_{i}s_{i}=\sum_{i=1}^{t} m'_{i}s_{i}\Leftrightarrow 
     \sum_{i=1}^{t} m_{i}s_{i}-\sum_{i=1}^{t} m'_{i}s_{i}=0 \Leftrightarrow \\
     \sum_{i=1}^{t} \delta_{i}s_{i}=0 
 \end{gather*}

 Taking $m, m'$ such that $m \not= m'$ means that $m$ is distinct from $m'$
 at least for one subsequence $m_{i}\not=m'_{i}$.\\
 So we can assume, without loss of generality, that $i=1$ is an index (or the
 only index) which $m$ and $m'$ differ on.\\

 So we can split the last summation in 2 parts, choosing $\delta_{1}s_{1}$ as the first element and $ \sum_{i=2}^{t} \delta_{i}s_{i}$ as second element:

 \begin{gather*}
         \delta_{1}s_{1}+\sum_{i=2}^{t} \delta_{i}s_{i}=0 \Leftrightarrow \\
         \delta_{1}s_{1}=-\sum_{i=2}^{t} \delta_{i}s_{i} \Leftrightarrow \\
         s_{1}=\frac{-\sum_{i=2}^{t} \delta_{i}s_{i}}{\delta_{1}}
     \end{gather*}

     and this means that when a collision happens $s_{1}$ must be exactly equal two the second member of the equation, which is an element of $\mathbb{F}$. But since every seed is chosen at random among $ \mathbb{F} $, what's the probability of picking the element $s_{1}$ which zeroes the above equation?

     This probability is just $\frac{1}{ \mathbb{F} } = \frac{1}{2^{n}} \in
     \negl{\lambda} $.

     \subsubsection{$\H$ with Galois fields elements and polynomials}

 \begin{construction}
     Take $ \mathbb{F}=GF(2^{n})$, a \textit{Galois field} of $2^{n}$
     elements.\\
     Let $m=(m_{1}, ..., m_{t}) \in \mathbb{F}^{t} $ and $s \leftarrow\$ 
     \mathbb{F}^{t} $. We state that 
     \[
         h_{s}(m)= \sum_{i=1}^{t}s^{i-1}m_{i}
        \]
 \end{construction}


\begin{exercise}
    Prove that this construction is \textbf{almost universal}.\\
    (possible proof: to be almost universal, looking at the definition,
    collisions with $m \not= m'$ must be negligible.\\
    So consider a collision as above: it must be true that  
    \[
     \sum_{i=1}^{t} m_{i}s^{i-1}=\sum_{i=1}^{t} m'_{i}s^{i-1} \Leftrightarrow 
     \sum_{i=1}^{t} m_{i}s^{i-1}-\sum_{i=1}^{t} m'_{i}s^{i-1}=0 \Leftrightarrow
     q_{m-m'}(s)=0
    \]
    How can we make a polynomial equal to 0? We have to find the \textbf{roots}
    of the polynomial, which we know are at most the \textbf{grade} of the polynomial . 
    So, the grade of this polynomial is $t-1$, and the probability of picking a
    root from $ \mathbb{F} $ as seed of $h_{s}(.)$ is 
    \[
        \P [s=root]=\frac{t-1}{2^{n}}   \in \negl{\lambda} 
    \] )
\end{exercise}
 