\mychapter{11}{Lesson 11}

% "The age of ULTRON" - wanted to keep that somewhere...
\section{Authenticated encryption (continued)}

Having proven \cpa-security of an authenticated encryption scheme in an \emph{Encrypt-then-Tag} mode, it remains to prove that it has the \textsc{auth} property. Before this, a new unforgeability definition is needed:

\begin{definition}
    Let $\Pi = (\textit{Tag}, \textit{Verify})$ be a \mac{} scheme. Then $\Pi$ is \textsc{eufcma}-secure iff it is \ufcma-secure, that is:
    \[
        \Pr[\cryptog{ufcma}(\lambda) = 1] \in \negl{\lambda}
    \]
    with the additional restriction that the tag $\phi^*$ of the forged message must be ``fresh'' itself.
\end{definition}

Note the small difference in security between \ufcma{} and \textsc{eufcma}.

\begin{theorem}
    Let $\Pi = (Enc, Dec, Tag, Verify)$ be an authenticated encryption scheme, composed by a \ske{} scheme $\Pi_1$ and a \mac{} scheme $\Pi_2$. If $\Pi_2$ is \textsc{eufcma}, then $\Pi$ has the \textsc{auth} property.
\end{theorem}

% AP190114: Need to phrase theorems differently
\begin{proof}
    The proof is analogous to the previous proof regarding the scheme's \cpa{} security. Suppose that $\Pi$ has not the \textsc{auth} property; then an adversary can use the distinguisher $\distinguisher^{\textsc{auth}}$ to successfully forge authenticated messages with fresh signatures against $\Pi_2$, as depicted in figure \ref{cryptoredux:ettauth}.
    
    \begin{cryptoredux}
        {ettauth}
        {Breaking authenticity of $\Pi_2$}
        {eufcma$_2$}
        {auth}

        \cseqbeginloop

        \return{}{$m$}{}
        \send{$c \pickUAR Enc(k_{1}, m)$}{$c$}{}
        \receive{$\phi \pickUAR Tag(k_{2}, c)$}{$\phi$}{}
        \invoke{}{$(c, \phi)$}{\shortstack[r]{
            $c \in C$ \\
            $\phi \in \Phi$
        }}

        \cseqendloop

        \cseqdelay

        \return{\shortstack[r]{
            $c^* \notin C$ \\
            $\phi^* \notin \Phi$
        }}{$(c^{*}, \phi^{*})$}{}
        \send{}{$(c^*, \phi^*)$}{\shortstack[l]{\textsc{Output 1 iff} \\ $\quad \textit{Verify}_k(c^*, \phi^*) = 1$}}

    \end{cryptoredux}

From $A^{auth}$ perspective, all the couples $(c_{i}, \phi_{i})$ received are made with the following schema:

\begin{equation*}
    c_{i} \in Enc(k_{1}, m \in \M) \wedge \phi_{i}\leftarrow\$ Enc(k_{2}, c_{i})
\end{equation*}

Since $\A^{auth}$ wins $Game^{auth}$, the challenge couple $(c^{}{*}, \phi^{*})$ which breaks $Game^{auth}$ will be produced to be decrypted as

\begin{equation*}
    Dec(k, (c^{*}, \phi^{*})) \rightarrow Dec(k_{1}, c^{*}) \in \M \wedge
    Dec(k_{2}, \phi^{*})=c^{*}
\end{equation*}

But if this happens , then $\A$ can use the same challenge couple of $\A^{auth}$ to win $Game^{ufcma}$, which is impossible.

It could happen that, for $c^{*}=c$ previously seen, $\phi^{*}$ is a new fresh tag, never seen before. Just in this case the $auth$ game would be valid because $(c^{*}, \phi^{*})$ would have never been seen before, but \textbf{not } the eufcma game, because $c^{*}$ was previously sent to the challenger.
\end{proof}
    
Now we want an ufcma secure scheme able to resist against message-tag challenge couples where the tag is fresh but the message has been already requested to the challenger.

\section{Pseudorandom permutations}
Pseudorandom permutations are like PRFs, but efficiently invertible.
Consider the following family of functions:
\[
    \F=\{ F_{k} : \{0,1\}^{n} \to \{0,1\}^{n} \}_{k \in \{0,1\}^{\lambda} }
\]

\begin{figure}[h!]
   \centering
   \sdinit{}
   \begin{tikzpicture}[scale=0.45]
      \sdbegin{}
      \newinst{A}{$ \A $}
      \newinst[4]{B}{$ C_{k\leftarrow\$\{0,1\}^{\lambda} 
      } $}

      \postlevel
      \mess{A}{$x$}{B}
      \node[anchor=west] at (mess to) {};
      \postlevel
      \mess{B}{$y$}{A}
      \node[anchor=west] at (mess from) {$y=F_{k}(x)$  };
\draw [->] (1.2,-3.3) to[out=240,in=110] (1.2,-1.6);
      \postlevel
      \mess{A}{$b'$}{B}
      \node[anchor=west] at (mess to) {  };


      \sdend{}
      \sdend{}
   \end{tikzpicture}
   \caption{$Real_{\F, \A}(\lambda)$}
\end{figure}

\begin{figure}[h!]
   \centering
   \sdinit{}
   \begin{tikzpicture}[scale=0.45]
      \sdbegin{}
      \newinst{A}{$ \A $}
      \newinst[4]{B}{$C_{P \leftarrow\$ \P(\lambda, n, n)} $}

      \postlevel
      \mess{A}{$x$}{B}
      \node[anchor=west] at (mess to) {};
      \postlevel
      \mess{B}{$y$}{A}
      \node[anchor=west] at (mess from) {$y=P(x)$  };
\draw [->] (1.2,-3.3) to[out=240,in=110] (1.2,-1.6);
      \postlevel
      \mess{A}{$b'$}{B}
      \node[anchor=west] at (mess to) {  };


      \sdend{}
      \sdend{}
   \end{tikzpicture}
   \caption{$Ideal_{\P, \A}(\lambda)$}
\end{figure}



and the two games are indistinguishable
\[
    Real_{\F, \A}(\lambda) \approx_{c} Ideal_{\F, \A}(\lambda)
\]
\subsection{Feistel Network}
Let $F:\{0,1\}^{n} \to \{0,1\}^{n} $ and $\psi_{F}$ the invertible function

\begin{gather*}
    \psi_{F}(\overbrace{x}^{\text{n bits}}, \overbrace{y}^{\text{n bits}})= (y,
    x \xor F(y))=(x', y')\\
    \psi^{-1}_{F}(\overbrace{x'}^{\text{n bits}}, \overbrace{y'}^{\text{n
    bits}})= (F(x') \xor y', x')=(x, y)
\end{gather*}

Is this function pseudorandom?

This is not pseudorandom, because the first n bits of the output of $\psi_{F}$
are always equal to $y$, while in a PRF the probability that, given two
different $(x, y)$ and $(x', y)$ in input, the first bits are equal is very
low.

\todo{FEISTEL IMAGE}

The $l-th$ level outputs something like 
\[
    \psi_{F}[l](x,y)=\psi_{F_{k_{l}}}(\psi_{F_{k_{l-1}}}(...(\psi_{F_{k_{1}}}(x,y))...))
\]
Two XORed rounds of this function don't create a PRP. In particular, imagine 2 queries 
$(x, y)$ and $(x', y)$ such that
\begin{equation*}
    \psi_{F, F'}(x,y) \xor  \psi_{F,F'}(x',y)=(x \xor F(y) \xor x' \xor F(y), \ldots)
\end{equation*}

Since for 2 random queries with the same $y$, the first member of the output is always equal to  $x \xor x'$ with probability 1, this XORed rounds cannot constitute a PRP (which, instead, for 2 queries with the same second member outputs a first member equal to $x \xor x'$ with negligible probability).

\begin{lemma}
    For every \textbf{unbounded}  distinguisher making $q \in poly(\lambda)$
    queries, the following are statistically close as long as $y_{1}, \ldots,
    y_{q}$ are \textbf{ \textit{y-nique} } , i.e. $ \forall i \not= j, y_{i}
    \not= y_{j}$
\end{lemma}
figures

