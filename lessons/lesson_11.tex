\mychapter{11}{Lesson 11}
\section{Authenticated encryption (Age of Ultron)}
Last time we proved CPA-security of $\Pi$. Today we will explore the
\textit{auth}  property.
Consider $\Pi$ as
    \begin{gather*}
        Enc:\{0,1\}^{\lambda}*\M \to \C \\
        Tag:\{0,1\}^{\lambda}*\C \to \Phi 
    \end{gather*}
   
    \begin{lemma}
        If $Tag(.,.)$ is \textbf{EUF-CMA} , then $\Pi$ has
    auth-property. 
    \end{lemma}
    
    What is \textbf{EUF-CMA} ?\\ 
    It's a property similar to \textbf{uf-cma} ,
    but now I want that the challenge message $(m^{*}, \phi^{*})$ is made by a
    fresh $m^{*}$ and a valid \textbf{fresh} $\phi^{*}$.\\ 
    The difference is that in ufcma we didn't care about the freshness of
    $\phi^{*}$.

\begin{proof}
    Suppose $\Pi$ has not the \textit{auth}  property.\\ 
    So we have an $\A'$ which can win the \textbf{auth} challenge of $\Pi$.\\
    On the other hand, we have a $\Pi_{2}$ schema which uses an \textbf{euf-cma}
    $Tag(.,.)$ function.\\
    So, by reduction, we show that 
    

\begin{figure}[h!]
   \centering
   \sdinit{}
   \begin{tikzpicture}[scale=0.7]
      \sdbegin{}
      \newinst{A}{$ \A^{auth} $}
      \newinst[4]{B}{$ \A_{k \leftarrow\$ \{0,1\}^{\lambda} 
      } $}
      \newinst[4]{C}{$\C^{eufcma}_{k_{2} \leftarrow\$ \{0,1\}^{\lambda} 
      }$ }

      \postlevel
      \mess{A}{$m$}{B}
      \node[anchor=west] at (mess to) {  };
      \postlevel
      \mess{B}{$c \leftarrow\$ Enc(k_{1}, m)$}{C}
      \node[anchor=west] at (mess to) {};
      \postlevel
      \mess{C}{$\phi$}{B}
      \node[anchor=west] at (mess from) {$\phi \leftarrow\$ Tag(k_{2}, c)$};
      \postlevel
      \mess{B}{$(c, \phi)$}{A}
      \node[anchor=west] at (mess to) {  };
\draw [->] (1.2,-5.9) to[out=240,in=110] (1.2,-1.3);

      \postlevel
      \mess{A}{$(c^{*}, \phi^{*})$}{B}
      \node[anchor=west] at (mess to) {  };
      \postlevel
      \mess{B}{$c^{*}, \phi^{*}$}{C}
      \node[anchor=west] at (mess to) { };

      \sdend{}
      \sdend{}
   \end{tikzpicture}
\end{figure}
The game returns 1 IFF
\[
    c^{*} \leftarrow\$ Enc(k_{1}, . ) \wedge \phi^{*} \leftarrow\$ Tag(k_{2}, c^{*}) \wedge (c^{*}, \phi) \not\in \{ (c, \phi ) \} 
\]

From $A^{auth}$ perspective, all the couples $(c_{i}, \phi_{i})$ received are
made with the following schema:
\[
    c_{i} \in Enc(k_{1}, m \in \M) \wedge \phi_{i}\leftarrow\$ Enc(k_{2}, c_{i})
\]
Since $\A^{auth}$ wins $Game^{auth}$, the challenge couple $(c^{}{*}, \phi^{*})$
which breaks $Game^{auth}$ will be produced to be decrypted as
\[
    Dec(k, (c^{*}, \phi^{*})) \rightarrow Dec(k_{1}, c^{*}) \in \M \wedge
    Dec(k_{2}, \phi^{*})=c^{*}
\]
But if this happens , then $\A$ can use the same challenge couple of $\A^{auth}$
to win $Game^{ufcma}$, which is impossible.\\

It could happen that, for $c^{*}=c$ previously seen, $\phi^{*}$ is a new
fresh tag, never seen before.
Just in this case the $auth$ game would be valid because $(c^{*}, \phi^{*})$ would
have never been seen before, but \textbf{not } the eufcma game, because $c^{*}$
was previously sent to the challenger.\\
\end{proof}
    
Now we want an ufcma secure scheme able to resist against message-tag challenge
couples where the tag is fresh but the message has been already requested to the
challenger.\\
\section{Pseudorandom permutations}
Pseudorandom permutations are like PRFs, but efficiently invertible.
Consider the following family of functions:
\[
    \F=\{ F_{k} : \{0,1\}^{n} \to \{0,1\}^{n} \}_{k \in \{0,1\}^{\lambda} }
\]

\begin{figure}[h!]
   \centering
   \sdinit{}
   \begin{tikzpicture}[scale=0.45]
      \sdbegin{}
      \newinst{A}{$ \A $}
      \newinst[4]{B}{$ C_{k\leftarrow\$\{0,1\}^{\lambda} 
      } $}

      \postlevel
      \mess{A}{$x$}{B}
      \node[anchor=west] at (mess to) {};
      \postlevel
      \mess{B}{$y$}{A}
      \node[anchor=west] at (mess from) {$y=F_{k}(x)$  };
\draw [->] (1.2,-3.3) to[out=240,in=110] (1.2,-1.6);
      \postlevel
      \mess{A}{$b'$}{B}
      \node[anchor=west] at (mess to) {  };


      \sdend{}
      \sdend{}
   \end{tikzpicture}
   \caption{$Real_{\F, \A}(\lambda)$}
\end{figure}

\begin{figure}[h!]
   \centering
   \sdinit{}
   \begin{tikzpicture}[scale=0.45]
      \sdbegin{}
      \newinst{A}{$ \A $}
      \newinst[4]{B}{$C_{P \leftarrow\$ \P(\lambda, n, n)} $}

      \postlevel
      \mess{A}{$x$}{B}
      \node[anchor=west] at (mess to) {};
      \postlevel
      \mess{B}{$y$}{A}
      \node[anchor=west] at (mess from) {$y=P(x)$  };
\draw [->] (1.2,-3.3) to[out=240,in=110] (1.2,-1.6);
      \postlevel
      \mess{A}{$b'$}{B}
      \node[anchor=west] at (mess to) {  };


      \sdend{}
      \sdend{}
   \end{tikzpicture}
   \caption{$Ideal_{\P, \A}(\lambda)$}
\end{figure}



and the two games are indistinguishable
\[
    Real_{\F, \A}(\lambda) \approx_{c} Ideal_{\F, \A}(\lambda)
\]
\subsection{Feistel Network}
Let $F:\{0,1\}^{n} \to \{0,1\}^{n} $ and $\psi_{F}$ the invertible function
\begin{gather*}
    \psi_{F}(\overbrace{x}^{\text{n bits}}, \overbrace{y}^{\text{n bits}})= (y,
    x \xor F(y))=(x', y')\\
    \psi^{-1}_{F}(\overbrace{x'}^{\text{n bits}}, \overbrace{y'}^{\text{n
    bits}})= (F(x') \xor y', x')=(x, y)
\end{gather*}
Is this function pseudorandom?\\
This is not pseudorandom, because the first n bits of the output of $\psi_{F}$
are always equal to $y$, while in a PRF the probability that, given two
different $(x, y)$ and $(x', y)$ in input, the first bits are equal is very
low.\\

===========================================================================\\
FEISTEL IMAGE\\
===========================================================================\\
The $l-th$ level outputs something like 
\[
    \psi_{F}[l](x,y)=\psi_{F_{k_{l}}}(\psi_{F_{k_{l-1}}}(...(\psi_{F_{k_{1}}}(x,y))...))
\]
Two XORed rounds of this function don't create a PRP. In particular, imagine 2 queries 
$(x, y)$ and $(x', y)$ such that
\[
    \psi_{F, F'}(x,y) \xor  \psi_{F,F'}(x',y)=(x \xor F(y) \xor x' \xor F(y),
    \ldots)
\]
Since for 2 random queries with the same $y$, the first member of the output is
always equal to  $x \xor x'$ with probability 1, this XORed rounds cannot
constitute a PRP (which, instead, for 2 queries with the same second member
outputs a first member equal to $x \xor x'$ with negligible probability).\\

\begin{lemma}
    For every \textbf{unbounded}  distinguisher making $q \in poly(\lambda)$
    queries, the following are statistically close as long as $y_{1}, \ldots,
    y_{q}$ are \textbf{ \textit{y-nique} } , i.e. $ \forall i \not= j, y_{i}
    \not= y_{j}$
\end{lemma}
figures

