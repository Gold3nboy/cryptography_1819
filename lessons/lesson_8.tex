\mychapter{8}{Lesson 8}

\section{Domain extension}

Up until now, encryption has been dealt with messages of fixed size around a polynomial function to $\lambda$. How to deal with messages with aritrary size? Setting a maximum bound to message length seems impractical, both for waste reasons when messages are too short, and for practicality when messages eventually get too long. The solution takes the form of a ``block-cipher'', where a message of a given size is split into equally-sized blocks, and then encrypted using a fixed-size encryption scheme. Various instances of this technique, called \emph{modes}, have been devised.

\subsection{Electronic Codebook mode}

The operation of \textsc{ecb}-mode is straightforward: Given a message split into blocks $(m_1, \dots, m_t)$, apply the scheme's encryption routine to each block, as shown in figure \ref{fig:ecb}:
\[
    c_i = F_k(r) \xor m_i \quad \forall i \in \{0, \dots, t\}
\]
Decryption is trivially implemented by \textsc{xor}-ing the ciphered blocks with $F_k(r)$.


\begin{figure}[ht]
    \centering

    \tikzstyle{box}  = [draw, minimum size=2em]
    \tikzstyle{init} = [pin edge={to-,thin,black}, pin distance = 0.3cm]

    \begin{tikzpicture}[node distance = 1.2cm, auto, >=latex']

        \node (r1) {$r$};
        \node (r2) [right of = r1, node distance = 2cm] {$r$};
        \node (rd) [right of = r2, node distance = 2cm] {$\dots$};
        \node (rt) [right of = rd, node distance = 2cm] {$r$};

        \node (f1) [box, below of = r1] {$F_k$};
        \node (f2) [box, below of = r2] {$F_k$};
        \node (fd) [below of = rd] {$\dots$};
        \node (ft) [box, below of = rt] {$F_k$};
        
        \node (x1) [below of = f1, pin={[init]left:$m_1$}] {$\xor$};
        \node (x2) [below of = f2, pin={[init]left:$m_2$}] {$\xor$};
        \node (xd) [below of = fd] {$\dots$};
        \node (xt) [below of = ft, pin={[init]left:$m_t$}] {$\xor$};
        
        \node (c1) [below of = x1] {$c_1$};
        \node (c2) [below of = x2] {$c_2$};
        \node (cd) [below of = xd] {$\dots$};
        \node (ct) [below of = xt] {$c_t$};

        \path[->] (r1) edge (f1);
        \path[->] (r2) edge (f2);
        \path[->] (rt) edge (ft);

        \path[->] (f1) edge (x1);
        \path[->] (f2) edge (x2);
        \path[->] (ft) edge (xt);
        
        \path[->] (x1) edge (c1);
        \path[->] (x2) edge (c2);
        \path[->] (xt) edge (ct);

    \end{tikzpicture}
    \caption{\textsc{ecb}-mode block-cipher in action, using a \prf{} as the encryption routine}
    \label{fig:ecb}
\end{figure}

This approach has the advantage of being completely parallelizable, as each block can clearly be encrypted separately; however there is a dangerous flaw in being not \cpa-secure, even when using a \prf-based encryption scheme. To understand why, observe that random nonces for ciphertext randomization are chosen per-message; this means the encryption of message blocks become deterministic in the message scope, enabling an adversary to attack the scheme within a single plaintext. It is sufficient to choose an all-0 or all-1 message to realize that all its blocks would encrypt to the same ciphered block.

\subsection{Cipher block chaining mode (\textsc{cbc})}

% AP190105: The explanation anticipates PRPs, which may be a problem. Should be reviewed at later times...
This mode serializes block encryption by using the preceding ciphered block in the formula:
\[
    c_i = P_k(r) \xor m_i \quad \forall i \in \{0, \dots, t\}
\]
This time, a \emph{pseudorandom permutation}(\prp) is used instead of a \prf; they will be discussed later on. The diagram in figure \ref{fig:cbc} shows a general view of \textsc{cbc}-mode's operation. The decryption process is analogous but in a reversed fashion, by computing the preimage of a ciphered block and \textsc{xor}-ing it with the preceding ciphered block:
\[
    m_i = P_k^{-1}(c_i) \xor c_{i-1}
\]

\begin{figure}[ht]
    \centering

    \tikzstyle{box}  = [draw, minimum size=2em]
    \tikzstyle{init} = [pin edge={to-,thin,black}, pin distance = 0.3cm]

    \begin{tikzpicture}[node distance = 1cm, auto, >=latex']

        \node (r1) {$m_1$};
        \node (r2) [right of = r1, node distance = 2cm] {$m_2$};
        \node (rd) [right of = r2, node distance = 2cm] {$\dots$};
        \node (rt) [right of = rd, node distance = 2cm] {$m_t$};
        
        \node (x1) [below of = r1] {$\xor$};
        \node (x2) [below of = r2] {$\xor$};
        \node (xd) [below of = rd] {$\dots$};
        \node (xt) [below of = rt] {$\xor$};

        \node (f1) [box, below of = x1] {$P_k$};
        \node (f2) [box, below of = x2] {$P_k$};
        \node (fd) [below of = xd] {$\dots$};
        \node (ft) [box, below of = xt] {$P_k$};
        
        \node (c1) [below of = f1] {$c_1$};
        \node (c2) [below of = f2] {$c_2$};
        \node (cd) [below of = fd] {$\dots$};
        \node (ct) [below of = ft] {$c_t$};

        
        \path[->] (r1) edge (x1);
        \path[->] (r2) edge (x2);
        \path[->] (rt) edge (xt);

        \path[->] (x1) edge (f1);
        \path[->] (x2) edge (f2);
        \path[->] (xt) edge (ft);
        
        \path[->] (f1) edge (c1);
        \path[->] (f2) edge (c2);
        \path[->] (ft) edge (ct);

        \node (r) [left of = x1, node distance = 2cm] {$r$};
        \node (c0) [left of = c1, node distance = 2cm] {$c_0$};
        \path[->] (r) edge (c0);
        
        \draw[->] (c0) -- +(1cm, 0) -- +(1cm, 2cm) -- (x1);
        \draw[->] (c1) -- +(1cm, 0) -- +(1cm, 2cm) -- (x2);
        \draw[->] (c2) -- +(1cm, 0) -- +(1cm, 2cm) -- (xd);
        \draw[->] (cd) -- +(1cm, 0) -- +(1cm, 2cm) -- (xt);

    \end{tikzpicture}
    \caption{\textsc{cbc}-mode block-cipher in action, using a \prp{} as the encryption routine}
    \label{fig:cbc}
\end{figure}

\subsection{Counter mode}

This mode closely resembles \textsc{ecb}-mode but uses a ``rolling'' nonce instead of a static one, as shown in figure \ref{fig:ctr}. At each successive block, the nonce is incremented by 1 and then used in a single block encryption. Since the nonce is in $\{0,1\}^n$, the increment is done modulo $2^{n}$ so that the value will wrap around to 0 it it ever overflows. Decryption is analogous.

\begin{figure}[ht]
    \centering

    \tikzstyle{box}  = [draw, minimum size=2em]
    \tikzstyle{init} = [pin edge={to-,thin,black}, pin distance = 0.3cm]

    \begin{tikzpicture}[node distance = 1cm, auto, >=latex']

        \node (r1) {$r$};
        \node (r2) [right of = r1, node distance = 2cm] {$r + 1$};
        \node (rd) [right of = r2, node distance = 2cm] {$\dots$};
        \node (rt) [right of = rd, node distance = 2cm] {$r + t - 1$};

        \node (f1) [box, below of = r1] {$F_k$};
        \node (f2) [box, below of = r2] {$F_k$};
        \node (fd) [below of = rd] {$\dots$};
        \node (ft) [box, below of = rt] {$F_k$};
        
        \node (x1) [below of = f1, pin={[init]left:$m_1$}] {$\xor$};
        \node (x2) [below of = f2, pin={[init]left:$m_2$}] {$\xor$};
        \node (xd) [below of = fd] {$\dots$};
        \node (xt) [below of = ft, pin={[init]left:$m_t$}] {$\xor$};
        
        \node (c1) [below of = x1] {$c_1$};
        \node (c2) [below of = x2] {$c_2$};
        \node (cd) [below of = xd] {$\dots$};
        \node (ct) [below of = xt] {$c_t$};
        
        \node (c0) [left of = c1, node distance = 2cm] {$c_0$};
        
        \path[->] (r1) edge (r2);
        \path[->] (r2) edge (rd);
        \path[->] (rd) edge (rt);

        \path[->] (r1) edge (f1);
        \path[->] (r2) edge (f2);
        \path[->] (rt) edge (ft);

        \path[->] (f1) edge (x1);
        \path[->] (f2) edge (x2);
        \path[->] (ft) edge (xt);
        
        \path[->] (x1) edge (c1);
        \path[->] (x2) edge (c2);
        \path[->] (xt) edge (ct);

        \draw[->] (r1) -- (-2cm, 0) -- (c0);

    \end{tikzpicture}
    \caption{Counter-mode block-cipher in action, using a \prf{} as the encryption routine}
    \label{fig:ctr}
\end{figure}

This apparently innocuous change to \textsc{ebc} is enough to ensure \cpa-security, at the cost of perfect parallelization.

\begin{theorem}
    Assume $F_k$ is a \prf, then the counter-mode block cipher (\textsc{ctr}) is \cpa-secure for variable length messages\footnotemark.
\end{theorem}
\footnotetext{\emph{Variable length messages} exactly means every message $m = (m_1, \dots, m_t)$ is made of $t$ blocks, and $t$ can change from any message to a different one.}

\begin{proof}
    Figure \ref{cryptogame:ctrcpa} models a \cpa{} attack to a counter-mode block-cipher. The proof will proceed by hybrid argument starting from this game, therefore the statement to verify will be $\cryptog{cpa}[\textsc{ctr}](\lambda, 0) \compindist \cryptog{cpa}[\textsc{ctr}](\lambda, 1)$.

    \begin{cryptogame}
        {ctrcpa}
        {A chosen plaintext attack to counter-mode block-cipher}
        {cpa}
        
        \cseqbeginloop

        \send{}{$m = (m_1, \dots, m_t)$}{}
        \receive{\shortstack[l]{
            $r = c_0 \pickUAR \{0, 1\}^n$ \\
            $c_i = F_k(r + i - 1) \xor m_i \quad \forall i$
        }}
        {$c = (c_0, c_1, \dots, c_t)$}{}

        \cseqendloop

        \cseqdelay
        
        \send{$|\mu_0| = |\mu_1| \in \mathcal{M}$}{$\mu_0, \mu_1$}{}
        \receive{\shortstack[l]{
            $\rho = \gamma_0 \pickUAR \{0, 1\}^n$ \\
            $b \pickUAR \{0, 1\}$ \\
            $(\gamma_b)_i = F_k(\rho + i - 1) \xor (\mu_b)_i \quad \forall i$
        }}
        {$\gamma_b = (\gamma_0, \gamma_1, \dots, \gamma_t)$}{}

        \cseqdelay
        
        \cseqbeginloop

        \send{}{$m = (m_1, \dots, m_t)$}{}
        \receive{\shortstack[l]{
            $r = c_0 \pickUAR \{0, 1\}^n$ \\
            $c_i = F_k(r + i - 1) \xor m_i \quad \forall i$
        }}
        {$c = (c_0, c_1, \dots, c_t)$}{}

        \cseqendloop

        \cseqdelay
        
        \send{}{$b'$}{\textsc{Output 1 iff} $b'=b$}

    \end{cryptogame}

    Define the two hybrid games from the original \cpa{} game as follows:

    \begin{itemize}
        \item $\hybridg{1}[\textsc{ctr}](\lambda, b)$: A random function $R$ is chosen \uar{} from $\R(\lambda, n, n)$ at the beginning of the game, and is used in place of $F_k$ in all block encryptions;
        \item $\hybridg{2}[\textsc{ctr}](\lambda, b)$: The challenger will pick random values from $\{0, 1\}^n$ as ciphered blocks, disregarding any encryption routine.
    \end{itemize}

    \begin{lemma}
        $\cryptog{cpa}[\textsc{ctr}](\lambda, b) \compindist \hybridg{1}[\textsc{ctr}](\lambda, b) \quad \forall b \in \{0, 1\}$
    \end{lemma}

    \begin{proof} The proof is left as exercise.
        
        \emph{Hint}: Since the original game and the first hybrid are very similar, we can use a distinguisher which plays the \cpa-game; since this is a lemma, our goal in the reduction is to break the precondition contained in the theorem statement.
    \end{proof}

    \begin{lemma}
        $\hybridg{1}[\textsc{ctr}](\lambda, b) \compindist \hybridg{2}[\textsc{ctr}](\lambda, b) \quad \forall b \in \{0, 1\}$
    \end{lemma}

    \begin{proof}
    
        Since $m_i$ doesn't affect the distribution of the result at all, for any $i$, if $R(r^{*})$ behaves like a true random extractor, then the two hybrids are indistinguishable in the general case ($R(r + i) \xor m_i \approx R(r + i)$). However, there is a sneaky issue: if in both games it happens that a given nonce $r_i$ is used in both one query encryption and the challenge mesage encryption at any step, the subsequent encrypted blocks will be completely random in the second hybrid, whereas in the first hybrid the function's images, albeit random, become predictable, enabling a \cpa.

        Nevertheless, it can be proved that these ``collisions'' happen with negligible probability within $\hybridg{1}[\textsc{ctr}]$. Let:
        \begin{itemize}
            \item $q$ = number of encryption queries in a game run
            \item $t_i$ = number of blocks for the $i$-th query
            \item $\tau$ = number of blocks for the challenge ciphertext
            \item \textsc{Overlap} event: $\exists i, j, \iota : r_i + j = \rho + \iota$
        \end{itemize}

        The \textsc{Overlap} event exaclty models our problematic scenario. Now it suffices to show that it occurs negligibly. For simplicity, assume the involved messages are of same length, that is $t_i = \tau =: t$. Denote with $\textsc{Overlap}_i$ to be the event that the $i$-th query overlaps the challenge sequence as specified above.

        Fix some $\rho$. One can see that $\textsc{Overlap}_i$ happens if:
        \[
            \rho - t + 1 \leq r_i \leq \rho + t - 1
        \]
        which means that $r_i$ should be chosen \emph{at least} in a way that:
        \begin{itemize}
            \item the sequence $\rho, \dots, \rho + t - 1$ comes before the sequence $r_i, \dots, r_i + t - 1$, and they overlap just for the last element $\rho + t - 1 = r_i$ or
            \item the sequence $r_i, \dots, r_i + t - 1$ comes before the sequence the sequence $\rho, \dots, \rho + t - 1$, and they overlap just for the last element $r_i + t - 1 = \rho$.
        \end{itemize}

        Then:

        \begin{align*}
            \Pr[\textsc{Overlap}_i] &= \frac{(\rho + t - 1) - (\rho - t + 1) + 1}{2^n} \\
            &= \frac{2t-1}{2^n} \\
            \Pr[\textsc{Overlap}] &\leq \sum_{i = 1}^{t} \Pr[\textsc{Overlap}_i] \\
            &\leq 2 \frac{t^2}{2^n} \in \negl{\lambda} 
        \end{align*}
            
        which proves that our collision scenario happens with negligible pobability, thus the two hybrids are indistinguishable.

    \end{proof}
    
    Having proven the indistinguishability between the hybrids, the conclusion is reached:
    \[
        \cryptog{cpa}[\textsc{ctr}](\lambda, 0) \compindist \cryptog{cpa}[\textsc{ctr}](\lambda, 1)
    \]
    
\end{proof}
