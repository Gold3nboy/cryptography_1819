\mychapter{15}{Lesson 15}
\section{Public key encryption recap} 

$\mbox{Game}_{\Pi, \mathcal{A}}^{pke-cca}$:

\begin{figure}[h!]
    \centering
    \sdinit{}
    \begin{tikzpicture}
        \sdbegin{}

        \newinst{A}{$ \mathcal{A} $}     % Adversary
        % NOTE: the bracketed value expresses the distance between the two entities
        \newinst[3]{C}{$ C^{\textsc{pke-cca}} $}  % Challenger

        % BEGIN

        % Challenger publishes pk
        \mess{C}{$pk$}{A}
        \node[anchor=west] at (mess from) {
            \small{$(pk, sk) \pickUAR \mathcal{KG}en(1^\lambda)$}
        };
      
        \postlevel

        % Adversary queries for a decryption
        \mess{A}{$c'$}{C}
        \node[anchor=west] at (mess to) {};
        % Challenger responds with decrypted value
        \mess{C}{$m'$}{A}
        \node[anchor=west] at (mess to) {};
        % Adversary can query polynomially-many times
        \draw [->] (0.9,-3.3) to[out=200,in=150] (0.9,-2.4);
        
        \postlevel
        
        % Adversary starts the challenge by sending two messages
        \mess{A}{$m_0, m_1$}{C}
        \node[anchor=west] at (mess to) {};
        % Challenger flips a coin, encrypts the chosen message and sends the cipherext
        \mess{C}{$c$}{A}
        \node[anchor=west] at (mess from) { \shortstack[l]{
            $ b \pickUAR \{0, 1\} $ \\
            $ c \pickUAR Enc(pk, m_b) $
        }};
      
        \postlevel
      
        % Adversary can still query for other decryptions
        \mess{A}{$c'$}{C}
        \node[anchor=west] at (mess to) {};
        \mess{C}{$m'$}{A}
        \node[anchor=west] at (mess to) {};
        \draw [->] (0.9,-6.9) to[out=200,in=150] (0.9,-6.0);
      
        \postlevel

        % Adversary guesses which message has been encrypted
        \mess{A}{$b'$}{C}
        \node[anchor=west] at (mess to) {
            \small{Output 1 iff $b' = b$}
        };
      
        \sdend{}
    \end{tikzpicture}
    \caption{CCA on a PKE scheme}
    \label{fig:game_pke-cca}
\end{figure}

(Reminder): Encryptions are made like this: $Enc(k, m) = (r, F_k(r) \oplus m), r \simeq \mathcal{U}nif(\{0, 1\}^\lambda)$.

Every time an encryption is made, a fresh value $r$ is picked UAR.

\pagebreak
\subsection{Trapdoor permutation}

A TDP(TrapDoor Permutation) is a OWP family structured has these features:

\begin{itemize}
    \item A key pair is chosen UAR by a key generator algorithm: $(pk, sk) \pickUAR \mathcal{KG}en(1^\lambda)$
    \item There is a function family $f_{pk} \subseteq (V_{pk} \to V_{pk})$ such that:
    \begin{itemize}
        \item Computing $f_{pk}$ is efficient
        \item Sampling from the domain ($x \pickUAR V_{pk}$) is efficient
    \end{itemize}
    \item There is an efficient function $g_{sk}$ that ``inverts'' $f_{pk}$ ($sk$ is the ``trapdoor''): 
    \begin{equation*}
        g(sk, f(pk, x)) = x
    \end{equation*}
    \item No efficient adversary is able to invert $f_{pk}$ wihout knowing $sk$
\end{itemize}

Note: Since $pk$ is public, any adversary gets the capability of encrypting messages for free, without requiring n external challenger/oracle!

Therefore, if left deterministic, a TDP is not CPA-secure.

%The idea is to reuse the message as part of the randomness, or more in detail, use its hardcore predicate. The resulting scheme would the look like this:
Here, in this scheme, we combine randomness and the notion of hardcore predicate $\mathfrak{hc}$:


\begin{itemize}
    \item $(pk, sk) \pickUAR \mathcal{KG}en(1^\lambda)$
    \item $r \pickUAR \Xi_{pk}$
    \item $c := Enc(pk, m) = (f_{pk}(r), \mathfrak{hc}(r) \oplus m)$
    \item Correctness: $Dec(sk, c) = \mathfrak{hc}(g_{sk}(c_1)) \oplus c_2$
    % Exercise: If obscure, do the petty algebra on paper
\end{itemize}

Theorem: if $(\mathcal{KG}en, f, g)$ is a TDP, and $\mathfrak{hc}$ is hardcore for $f$, then the above scheme is CPA-secure.

Proof: (Execrcise)
\todo{Apparently, the reduction here is not easy at all. Some hints are needed
to solve this exercise.}

\subsection{TDP examples}

One example stems form the factoring problem: let's look again at $\mathbb{Z}_n^\times$, where $n = pq$, $p, q \in \mathbb{P}$;

Theorem (Chinese remainder, or CRT): The following isomorphisms to $\mathbb{Z}_n^\times$ are true:

\begin{itemize}
    \item $\mathbb{Z}_n \simeq \mathbb{Z}_p \times \mathbb{Z}_q$
        \item $\mathbb{Z}_n^\times \simeq \mathbb{Z}_p^\times \times \mathbb{Z}_q^\times$
\end{itemize}

Note that the theorem is more general, and holds for any $p, q$ that are coprime.

How to use this theorem for constructing a PKE scheme:

Reminder (Euler's theorem):\review{ $\forall x \in \mathbb{Z}_n \implies
x^{\varphi(n)} = x \mod n$ \\
maybe the correct one is \\
$\forall x \in \mathbb{Z}_n \implies
x^{\varphi(n)} = 1 \mod n$
}

Reminder: $\forall p, q \in \mathbb{P} \implies \varphi(pq) = (p-1)(q-1)$

So let $a$ be the public key such that $\gcd(a, \varphi(n))=1$, then $\exists! b
\in \mathbb{Z}_n : ab = 1 \mod \varphi(n)$, $b$ will be our private key.

Define encryption as $f(a, m) = m^a \mod n$, and then decryption as $g(b, c) =
c^b \mod n$.

Observe that 
\[
g(b, f(a, m)) = (m^a)^b =m^{ab}=m^{k\varphi(n)+1}=(m^{\varphi(n)})^{k}m = m \mod n
\]
, because $ab = 1 \mod \varphi(n)$.

So we conjecture that the above is a valid TDP-based PKE scheme. This is actually called the ``RSA assumption'':

\begin{figure}[h!]
    \centering
    \sdinit{}
    \begin{tikzpicture}
        \sdbegin{}

        \newinst{A}{$ \mathcal{A} $}     % Adversary
        % NOTE: the bracketed value expresses the distance between the two entities
        \newinst[3]{C}{$ C^{\textsc{rsa}} $}  % Challenger

        % BEGIN

        % Adversary asks for the challenge
        \postlevel
        \mess{C}{$n, pk, m^{pk}$}{A}
        \node[anchor=west] at (mess from) { \shortstack[l]{
            $ b \pickUAR \{0, 1\} $ \\
            $ c \pickUAR Enc(pk, m_b) $
        }};

        % Adversary guesses the plaintext
        \postlevel
        \mess{A}{$m'$}{C}
        \node[anchor=west] at (mess to) {
            \small{Output 1 iff $m' = m$}
        };
      
        \sdend{}
    \end{tikzpicture}
    \caption{Depiction of the RSA assumption}
    \label{fig:game_pke-rsa}
\end{figure}

Relation to the factoring problem: $\textsc{rsa} \implies \textsc{fact}$

Proof: Given $p, q$, an adversary can compute $\varphi(n) = (p-1)(q-1)$, and then find the inverse of the public key in $\mathbb{Z}_{pq}^\times$.

It hasn't been proven that $\textsc{fact} \implies \textsc{rsa}$









\section{Textbook RSA}
This is an \underline{insecure} toy example of the more complex \textit{RSA} (Rivest Shamir Adleman) algorithm.
The key generation algorithm: $\KGen=\Gen RSA(1^{\lambda})$ outputs $P_k=(n,e)$ and $S_k=d$, then we have

\begin{gather*}
    Enc(P_k,m)=m^e\Mod{n}\\
    Dec(S_k,c)=c^d\Mod{n}
\end{gather*}

Since the output of Enc is deterministic this is \textbf{not CPA secure}! However it can be used with HARD-CORE Predicate.\\
Preprocess the message to add randomness:
$$\hat{m}=r||m \text{ where }r\leftarrow\mathdollar\{0,1\}^l$$
now Enc is not deterministic.\\
\textbf{Facts:}
\begin{enumerate}
    \item $l \in super(log(\lambda))$ otherwise it is possible to bruteforce in PPT.
        \review{\item If $m\in\{0,1\}$ then I can prove it CPA secure under RSA
        (just use standard TDP)}
    \item If $m$ is "in the middle" ($\{0,1\} \leq m \leq \{0,1\}^l$) RSA is believed to be secure and is \underline{standardized} (PKCS\#1,5)
        \review{ \item Still not CCA secure! counterexample?}
\end{enumerate}



\subsection{Trapdoor Permutation from Factoring}
Let's look at $f(x)=x^2\Mod{n}$ where $f: \Z_n^* \to \QR[n] (\subset \Z_n^*)$, this is not a permutation in general.\\
Now let's consider the Chinese Reminder Theorem (CRT) representation:

    \begin{gather*}
        x=(x_p,x_q) \rightarrow x_p\equiv x\Mod{p} , x_q\equiv x\Mod{q}\\
        f(x)=x^2\Mod{p}; x \from \$ \Z_p^*
    \end{gather*}

Since $Z_p^*$ is cyclic I can always write:

\begin{gather*}
    Z_p^*=\{g^0,g^1,g^2,\ldots,g^{\frac{p-1}{2}-1},g^{\frac{(p-1)}{2}},\ldots,g^{p-2} \}\\
    \QR[p]=\{g^0,g^2,g^4,\ldots,\overbrace{g^{p-3}}^{g^{\frac{p-1}{2}-1} in Z_p^*},\underbrace{g^0}_{g^{\frac{p-1}{2}}in Z_p^*},\ldots\}\\
    |\QR[p]|=\frac{p-1}{2}
\end{gather*}
Moreover, since $(g^{\frac{p-1}{2}})^2 \equiv 1 \Mod{p} $ and $g^{\frac{p-1}{2}}$
cannot be 1 (since $g^{0}\not=g^{\frac{p-1}{2}}\not=g^{p-1}$ ) but must be one
of the $p-1$ elements of $Z_{p}^{*}$, then
$g^{\frac{p-1}{2}} \equiv -1 \Mod{p}$.\\

Now it's possible to show that $f: \QR[p]\to \QR[p] $ is a permutation, and we
are going to show a method to invert it, aka $f^{-1}$.\\

Assume $p\equiv 3 \Mod{4}$
([*]$p=4t+3\Rightarrow t=\frac{p-3}{4}$), then squaring $Mod{p}$ is a permutation because, given
\underline{$y=x^2 \Mod{p}$} if I compute: \begin{gather*}
    (y^{t+1})^2=\underbrace{y^{2t+2}}_{\text{[*]
    }2t+2=\frac{p-3}{2}+2=\frac{p+1}{2}=\frac{p-1}{2}+1}=(x^2)^{\frac{p-1}{2}+1}=1x^2=x^2\\
    \implies x=\pm y^{t+1} \end{gather*}

But only 1 among the above $\pm y^{t+1}$ is a square, in particular only the
positive one.\\
Since we have that 
\[ 
    p=k4+3 \Rightarrow \frac{p-1}{2}=\frac{4k+2}{2}=2k+1
\]
so $\frac{p-1}{2}$ is odd.\\

Now, since we are considering just $\QR[p]=\{y \in \mathbb{Z}_{p}^{*} : \exists
x \in \mathbb{Z}_{p}^{*} x^{2}=y\}$ and we can write each $x \in
\mathbb{Z}_{p}^{*} $ as
$g^{z}$ for a $z \in \mathbb{Z}_{p}$, 
\[
    y=x^{2} \Leftrightarrow y=(g^{z})^{2}=g^{2z}
\]

So, $y=g^{z'} \in \QR[p] \Leftrightarrow z'$ is even. If $z'$ is odd, then $y \notin \QR[p] $.

Since $\frac{p-1}{2}$ is odd, then $g^{\frac{p-1}{2}} \not\in \QR[p]$, and since
it is possible to generate all of the other numbers with odd exponents 
\[
g^{odd}=g^{\frac{p-1}{2} \pm even}=g^{\frac{p-1}{2}}g^{ \pm even} \Rightarrow
-1(g^{\pm even})
\]
and $g$ powered to odd exponents will have this form.\\

From that, it's possible to state the following
\begin{lemma}
    $\forall z, z\in \QR[p] \implies  -z \notin \QR[p]$
\end{lemma}



\subsection{Rabin's Trapdoor permutation}

Now we study a one way function built on previous deductions about number theory and modular arithmetic.

The \textit{Rabin trapdoor permutation} is defined as 
\[
    f(x)=x^{2} \text{ mod } n
\]
where $n=p*q$ for primes $p,q=3 \text{mod} 4$.\\

We can observe that the image of this function is $\QR[n]$, a subset of $\mathbb{Z}^{*}_{n} $.

For \textbf{Chinese remainder theorem} it is possible to state that $f$ maps
as follows
\[
    x= (x_{p}, x_{q}) \mapsto (x^{2}_{p}, x^{2}_{q})
\]
since each element of $ \mathbb{Z}_n$ has always two different forms , in $
\mathbb{Z}_{p} $ and in $ \mathbb{Z}_{q} $.\\

So
\[
y \in  \mathbb{QR}_{n} \Leftrightarrow y_{p} \in \mathbb{QR}_{p} \wedge y_{q} \in
\mathbb{QR}_{q} 
\]

As before, the image of $f$ is exactly
\[
    \mathbb{QR}_{n} = \{ y: \exists x : y=x^{2} \text{ mod } n\}
\]

If we try to invert the function $f$, even without applying the previous
inversion algorithm, we easily note that among the 4 possible values
\[
    f^{-1}(y)=\{ (x_{p}, x_{q}),(-x_{p}, x_{q}),(x_{p},- x_{q})(-x_{p},- x_{q})\}
\]\label{les15:outoffour}
only 1 is a quadratic residue since we said, in the
last lemma, that only one out of
$-x_{k}, x_{k}$ is a quadratic residue for $k=q,p$.\\

Therefore, we have that the Rabin's TDP is a permutation in $\QR[n]$, and that the
cardinality of $ \mathbb{QR}_{n} $ is $\frac{|\mathbb{Z}^{*}_{n} |}{4}$.\\

Furthermore, with the following claim we can state that the Rabin cryptosystem
is OWF thanks to the hardness of factoring.

\begin{claim}
    Given $x, z$ such that $x^{2}\text{ mod} n \equiv z^{2} \text{ mod } n
    \equiv y$ mod $n$,
\[
    x\not= \pm z \Rightarrow \text{ we can factor } n
\]
\end{claim}
\begin{proof}
    Since $f^{-1}(y)$ has only one value out of four, $x\neq \pm z$ and
    $z\in \{(x_{p},x_{q}),(-x_{p},-x_{q})\}$ , then
    $x \in \{(x_{p},-x_{q}),(-x_{p},x_{q})\}$ and 
    \[
        x + z \in \{(0,2x_{q}), (2x_{p}, 0)\}
    \]

    Now assume $x + z = (2x_{p}, 0)$ without loss of generality, since the proof for the other case is the same.

    We have that $x+z \equiv 0$ mod $q$ and $x+z \not\equiv 0 $ mod $p$.

    But then $gcd( x+z , n)=q$, and we obtain $q$.
\end{proof}

\begin{theorem}
    Squaring mod $n$ (where $n$ is a \textit{bloom integer}\footnote{ a bloom
    integer $n$ is $n=p,q$ for $p,q=3 mod 4$, as the definition of Rabin's TDP})
    is a \textbf{trapdoor permutation} under factoring.
\end{theorem}
Since we have already shown tha Rabin's function is a permutation since it is
invertible, we have to show that Rabin's function is also OWF.

In other words
\[
    \text{Factoring is hard } \Rightarrow \text{ $f(x)$ is OWF , aka inverting
    it is hard}
\]
The following proof is by contraddiction.
\begin{proof}
    Assume that exists an adversary PPT who, given $y=x^{2} \text{ mod }
    n$, can find a $z \in \mathbb{Z}_{n} $ such that $z^{2} \text{ mod } n=y$
    but $z\neq \pm x$ .\\

    We can build the following reduction to show that $\A$ choses $x$:

\begin{figure}[ht]
   \centering
   \sdinit{}
   \begin{tikzpicture}
      \sdbegin{}
      \newinst{A}{$ \A^{Rabin} $}
      \newinst[2.5]{B}{$ \A_{fact} $}
      \newinst[2]{C}{$\C_{(p,q,n) \leftarrow\$ Genbloom \wedge p,q \equiv 3
      \text{ mod } 4}$}
      
            \postlevel
            \mess{C}{$n=pq$}{B}
            \node[anchor=west] at (mess to) {  };
      
      \postlevel
      \mess{B}{$n,y$}{A}
      \node[anchor=west] at (mess from) {\shortstack[l]{
      $ x \leftarrow\$ \mathbb{Z}^{*}_{n} $
      \\
      $ y=x^{2} $ mod $n$
      }  };

      \postlevel
      \mess{A}{$z$}{B}
      \node[anchor=west] at (mess to) {\shortstack[l]{
              $ f(z)=f(x) $ 
             }};


      \sdend{}
      \sdend{}
   \end{tikzpicture}
\end{figure}

Once obtained $z\neq \pm x$ which $z^{2}=y$ we can use \textbf{Claim 1}(just
summing $x$ and $z$ and analyzing the result) to factorize $n$ in
polytime.\\

But factorizing $n$ in polytime is not possible.
\end{proof}


