\mychapter{1}{Lesson 1} %180926

\section{Problems in Cryptography}

- Confidential communication 

\todo{Image of Alice,Bob, Eve in ske}

\textbf{Kerckhoffs's principle:} \underline{No security by obscurity.}

Security should only rely on the secrecy of the key. However sharing the key between two parties is a costly operation.

\textbf{Syntax:}
\begin{itemize}
    \item $\K = $ Key Space
    \item $\M = $ Message Space
    \item $\C = $ Ciphertext Space
    \item $Enc: \K \times \M \to \C$
    \item $Dec: \K \times \C \to \M$
\end{itemize}

\textbf{Correctness:}

$\forall m \in \M, \forall k \in \K : Dec(k, Enc(k,m))=m$

\begin{definition}
    Perfect Secrecy (SHANNON)

    Let M be any distribution over $\M$.

    Let K be uniform over $\K$.

    We say that $(Enc, Dec)$ is perfectly secret if $\forall M, \forall m \in \M, \forall c \in \C$.

    $$Pr[M=m]=Pr[M=m|C=c]$$
    Meaning that the ciphertext does not reveal anything about the message.
\end{definition}

\begin{theorem}
    Let M, K, C as above. The following are equivalent:
\end{theorem}

\begin{enumerate}[I]
    \item $Pr[M=m]=Pr[M=m|C=c]$
    \item M and C are INDEPENDENT
    \item $\forall m, m' \in \M, \forall c \in C$ $Pr[Enc(k,m)=c]=Pr[Enc(k,m')=c]$
\end{enumerate}

\begin{proof}

    $(I)\implies(II)$
    \begin{gather*}
        Pr[M=m \wedge C=c]=Pr[C=c]Pr[M=m|C=c]=Pr[C=c]Pr[M=m]   
    \end{gather*}
    They are independent.
    
    $(II)\implies(III)$
    Fix M, any $m\in M, c\in C$
    \begin{align*}
        &\Pr[\textsf{Enc}(K,m)=c]= \\
        =&\Pr[\textsf{Enc}(K,M)=c|M=m]= \\
        =&\Pr[C=c|M=m]=\Pr[C=c]   
    \end{align*}

    $(III)\implies(I)$
    
    \begin{align*}
        \Pr[C=c] =& \sum_{m'}Pr[C=c\wedge M=m'] & \\
        =& \sum_{m'}Pr[Enc(K,M)=c\wedge M=m'] & \\
        =& \sum_{m'}Pr[Enc(K,M)=c| M=m']Pr[M=m'] & \\
        =& \sum_{m'}Pr[Enc(K,m')=c]Pr[M=m'] & \text{(Using III)} \\
        =& \Pr[Enc(K,m)=c]\sum_{m'}Pr[M=m'] & \\
        =& \Pr[Enc(K,m)=c] & \text{(Using total prob.)} \\
        =& \Pr[Enc(K,M)=c|M=m] & \\
        =& \Pr[C=c|M=m] &
    \end{align*}

    Furthermore:

    % AP190126: Correctness on these parts isn't verified, I may have rewrote them wrong...
    \begin{align*}
        \Pr[C=c] = \Pr[C=c|M=m] \implies& \Pr[M=m] = \Pr[C=c|M=m] \\
        \implies& \Pr[M=m] = \frac{\Pr[C=c]\Pr[M=m|C=c]}{\Pr[C=c|M=m]} \\
        \implies& \Pr[M=m] = \Pr[M=m|C=c]
    \end{align*}

\end{proof}

\section{One Time Pad (OTP)}

$K=M=C=\{0,1\}^l$\\
$K\pickUAR \{0,1\}^l$
\begin{itemize}
    \item $Enc(k,m)=k\xor m=c$
    \item $Dec(k,c)=k\xor c=m$
\end{itemize}
\begin{theorem}
    \textbf{OTP} is perfectly secret
\end{theorem}
\begin{proof}
    Fix any $m,c \in \{0,1\}^l$:
    \begin{align*}
        \Pr[Enc(k,m)=c] &= \Pr[k\xor m=c] \\
        &= \Pr[k=c\xor m]=2^{-l} \text{(because k is uniform)} \\
        &= \Pr[Enc(k,m')=c] \forall m'    
    \end{align*}
    This satisfies (III)
\end{proof}

\textbf{Problems:}
\begin{enumerate}
    \item $|k|=|m|$ (the key must have the same length of the message)
    \item There must be one different key for each message (\textit{One Time}).\\Otherwise:\\
    $c_1=k\xor m_1; c_2=k \xor m_2 \implies c_1\xor c_2=m1\xor m_2$\\
    Not independent! (OTP is malleable\footnote{An encryption algorithm is "malleable" if it is possible to transform a ciphertext into another ciphertext which decrypts to a related plaintext. That is, given an encryption $c$ of a plaintext $m$, it is possible to generate another valid ciphertext $c'$, for a known $Enc$, $Dec$, without necessarily knowing or learning $m$. })
\end{enumerate}

\begin{theorem}
    In any perfectly secret SKE we must have $|\K|\geq |\M|$
\end{theorem}
\begin{proof}
    Take $M$ to be uniform over $\M$ and take $c$ s.t. $Pr[C=c]>0$.\\
    Consider $M'=\{Dec(K,c):k\in K\}$ (decryption of $c$ with every key) and assume $|\K|<|\M|$
    \begin{gather*}
        \implies |M'|\leq |K|\leq |M|\\
        \implies \exists m \in M \setminus M'\\
        \text{Now} Pr[M=m]=\frac{1}{|\M|}\\
        Pr[\underbrace{M=m}_{m\text{ is not inside }M'}|\overbrace{C=c}^{\text{By assumption all the c "creates" M'}}]=\emptyset
    \end{gather*}
    The above notion can be represented with the following picture:\\
    \def\firstcircle{(0,0) circle (1.5cm)}
    \def\secondcircle{(60:0) circle (0.9cm)}
    \begin{tikzpicture}
        \begin{scope}[shift={(3cm,-5cm)}, fill opacity=0.5]
            \fill[red] \firstcircle;
            \fill[green] \secondcircle;
            \draw \firstcircle node[label={[xshift=1.0cm, yshift=0.3cm]$M$}] { };
            \draw \secondcircle node[label={[xshift=1.2cm, yshift=-0.5cm]$m$}] {$M'$};
        \end{scope}
    \end{tikzpicture}
    Meaning that $c$ is not an encryption of $m$!
\end{proof}