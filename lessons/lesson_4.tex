\mychapter{4}{Lesson 4}

\section{Negligible function}

What is exactly a negligible function? Below here there is a possible interpretation of this notion:
\begin{quotation}
    ``In real life, we can just consider adversaries with limited computational power; even if every non-perfectly secure authentication scheme can resist to unbounded computational power, the true unbounded computational power doesn't exist at all. So, it's reasonable to consider just bounded adversaries.

    So consider a scheme $\Pi$ where the only attack against it is brute-force attack. We consider $\Pi$ to be secure if it cannot be broken by a brute-force attack in polynomial time.

    The idea of \textbf{negligible probability} encompasses this exact notion. In $\Pi$, let's say that we have a polynomial-bounded adversary. Brute force attack is not an option.

    But instead of brute force, the adversary can try (a polynomial number of) random values and hope to guess the right one. In this case, we define security using negligible functions: The probability of success has to be smaller than the reciprocal of any polynomial function.

    And this makes a lot of sense: if the success probability for an individual guess is a reciprocal of a polynomial function, then the adversary can try a polynomial amount of guesses and succeed with high probability. If the overall success rate is $\frac{1}{poly(\lambda)}$ then we consider this attempt a feasible attack to the scheme, which makes the latter insecure.

    So, we require that the success probability must be less than the reciprocal of every polynomial function. This way, even if the adversary tries $poly(\lambda)$ guesses, it will not be significant since it will only have tried: $\frac{poly(\lambda)}{superpoly(\lambda)}$\footnote{If we design a function hard for $superpoly(\lambda)$ possible attempts and the attacker completed $poly(\lambda)$ attempts, he has just $ \P [\frac{poly(\lambda)}{superpoly(\lambda)}] $ of finding the key to break the scheme}

    As $\lambda$ grows, the denominator grows far faster than the numerator and the success probability will not be significant\footnote{\href{https://crypto.stackexchange.com/questions/5832/what-exactly-is-a-negligible-and-non-negligible-function}{\textsf{``What exactly is a negligible (and non-negligible) function?'' --- Cryptography Stack Exchange}}}.''

\end{quotation}

\subsubsection{Definition}
Let $\nu : \mathbb{N} \to [0,1]$ be a function. Then it is deemed \textbf{negligible} iff:
\begin{equation*}
    \forall p(\lambda) \in poly(\lambda) \implies \nu(\lambda) \in o(\frac{1}{p(\lambda)})
\end{equation*}

\begin{exercise}
    Let $p(\lambda), p'(\lambda) \in poly(\lambda)$ and $\nu(\lambda), \nu'(\lambda) \in \negl\lambda$. Then prove the following:

    \begin{enumerate}
        \item $p(\lambda) \cdot p'( \lambda) \in  \poly(\lambda)$
        \item $\nu(\lambda) + \nu'(\lambda) \in \negl\lambda$
    \end{enumerate} 
\end{exercise}

\begin{solution}[1.1]

    \todo{Questa soluzione usa disuguaglianze deboli; per essere negligibile una funzione dev'essere strettamente minore di un polinomiale inverso. Da approfondire}

    We need to show that for any $c \in \mathbb{N} $, then there is $n_{0}$ such that $\forall n > n_{0} \implies h(n) \leq n^{-c}$.
    
    So, consider an arbitrary $c \in \mathbb{N}$. Then, since $c+1 \in \mathbb{N} $, and both $f$ and $g$ are negligible, there exists $n_{f}$ and $n_{g}$ such that:
    \begin{align*}
        \forall n \geq n_{f} &\implies f(n) \leq n^{-(c+1)} \\
        \forall n \geq n_{g} &\implies g(n) \leq n^{-(c+1)}
    \end{align*}

    Fix $n_{0} = \max(n_{f}, n_{g})$. Then, since $n \geq n_{0} \geq 2$, $\forall n \geq n_{0}$ we have:
    \begin{align*}
        h(n) &= f(n)+g(n) \\
        &\leq n^{-(c+1)} + n^{-(c+1)} \\
        &= 2n^{-(c+1)} \\
        &\leq n^{-c}
    \end{align*}
    Thus we conclude $h(n)$ is negligible.
\end{solution}


\section{One-Way Functions}

A One-Way Function (\textsc{owf}) is a function that is ``easy to compute'', but ``hard to invert''.

\begin{definition}    
    Let $f : \{0,1\}^{n(\lambda)} \to {0,1}^{n(\lambda)}$ be a function. Then it is a \textsc{owf} iff:

    \begin{equation}
        \forall\; \textsc{ppt}\; \A\;
        \exists\nu(\lambda) \in \negl{x} :
        \Pr \left[ \textsc{Game}_{f,A}^\textsc{owf} (\lambda) = 1 \right]
        \leq \nu(\lambda)
    \end{equation}

\end{definition}


\begin{cryptogame}{owfdef}{One-Way Function hardness}{owf}
    \receive{\shortstack[l]{
        $x \pickUAR \{0,1\}^{n}$ \\
        $y = f(x)$}}
    {$y$}
    {}

    \postlevel

    \send{}
    {$x'$}
    {\textsc{Output 1 iff} $f(x')=y$}

\end{cryptogame}

Do note that the game does not look for the equality $x' = x$, but rather for the equality of their images computed by $f$.

\pagebreak

\begin{exercise}
    \
    \begin{enumerate}
        \item Show that there exists an inefficient adversary that wins $\textsc{Game}^\textsc{owf}$ with probability 1
        \item Show that there exists an efficient adversary that wins $\textsc{Game}^\textsc{owf}$ with probability $2^{-n}$
    \end{enumerate}
\end{exercise}


\subsubsection{One-Way puzzles}
A one-way function can be thought as a function which is very efficient in generating puzzles that are very hard to solve. Furthermore, the person generating the puzzle can efficiently verify the validity of a solution when one is given.

For a given couple $(\mathcal{P}_\textsc{gen},\mathcal{P}_\textsc{ver})$ of a puzzle generator and a puzzle verifier, we have:

\begin{cryptogame}{owpuzzle}{The puzzle game}{puzzle}
    \receive{$(x, y) \pickUAR \mathcal{P}_\textsc{gen}$}
    {$y$}
    {}

    \postlevel

    \send{}
    {$x'$}
    {\textsc{Output} $\mathcal{P}_\textsc{ver}(x', y)$}

\end{cryptogame}

So, we can say that one-way puzzle is a problem in \textsc{np}, because the verifier enables efficient solution verification, but not in \textsc{p} because finding a solution from scratch is impractical by definition.

\subsection{Impagliazzo's Worlds}

Suppose to have Gauss, a genius child, and his professor. The professor gives to Gauss some mathematical problems, and Gauss wants to solve them all.

Imagine now that, if using one-way functions, the problem is $f(x)$, and its solution is $x$. According to Impagliazzo, we live in one of these possible worlds:
\begin{itemize}
    \item \textit{Algorithmica}: $\textsc{P} = \textsc{NP}$, meaning all efficiently verifiable problems are also efficiently solvable. 
    
    The professor can try as hard as possible to create a hard scheme, but he won't succeed because Gauss will always be able to efficiently break it using the verification procedure to compute the solution

    \item \textit{Heuristica}: \textsc{NP} problems are hard to solve in the worst case but easy on average. 
    
    The professor, with some effort, can create a game difficult enough, but Gauss will solve it anyway; here there are some problems that the professor cannot find a solution to

    \item \textit{Pessiland}: \textsc{NP} problems are hard on average but no one-way functions exist
    
    \item \textit{Minicrypt}: One-way functions exist but public-key cryptography is impractical
    
    \item \textit{Cryptomania}: Public-key cryptography is possible: two parties can exchange secret messages over open channels
\end{itemize}
    
\section{Computational Indistinguishability}

Distribution ensembles $X = \{X_{\lambda \in \mathbb{N}}\}$ and $Y = \{Y_{\lambda \in \mathbb{N}}\}$ are distribution sequences.

\begin{definition}
    $X$ and $Y$ are \textbf{computationally indistinguishable} $(X \approx_{c} Y)$ if $\forall\; \textsc{ppt}\; D, \exists \nu(\lambda) \in \negl{\lambda}$ such that:
    \begin{equation*}
        \left| \Pr[D(X_\lambda) = 1] - \Pr[D(Y_\lambda) = 1] \right| \leq \nu(\lambda) 
    \end{equation*}
\end{definition}

\todo{AP181129-2344: There may be room for improvement, but I like how it's worded: it puts some unusual perspective into the cryptographic game, and it could be a good thing since it closely precedes our first reduction, and the whole hybrid argument mish-mash.}

Suppose we have this mental game: a Distinguisher $D$ receives the value $z$. This value has been chosen by me, the Challenger, among $X_{\lambda}$ and $Y_{\lambda}$, and the Distinguisher has to \textit{distinguish} which was the source of $z$. What does this formula mean?

This formula means that, fixed 1 as one of the sources, the \textit{probability} that D says ``1!'' when I pick $z$ from $X_{\lambda}$ is \textbf{not so far} from the \textit{probability} that D says ``1!'' when I pick $z$ from $Y_{\lambda}$.

So, this means that, when this property is verified by two random variables, there isn't too much \textit{difference} between the two variables in terms of exposed information (reachable by D), otherwise the distance between the two probabilities should be much more than a \textit{negligible} quantity.

What's the deep meaning of this formula? This is something to do.

\begin{lemma}\label{lem:tria}
    $\forall\; \textsc{ppt}\; f,\; X \compindist Y \implies f(x) \compindist f(y)$
\end{lemma}

\begin{proof}
    We want to show that $f(x) \compindist f(y)$. So, let's suppose this property is not true.

    Assume $\exists\; \textsc{ppt}\; f, A$ and some $p'(\lambda) \in \poly(\lambda) $ such that
    \begin{equation}
        \left| \Pr[D'(f(x)) = 1] - \Pr[D'(f(y)) = 1] \right| \geq \frac{1}{p'(\lambda)}
    \end{equation}.

    \begin{cryptogame}{fdistin}{A distinguisher for $f$}{$f$-\textsc{dist}}

        % Adversary asks for the challenge. Challenger chooses evenly between X and Y, and samples a value; then it sends that value's image by f to the adversary
        \receive{\shortstack[l]{
            $z_0 \pickUAR X$ \\
            $z_1 \pickUAR Y$ \\
            $b \pickUAR \{0, 1\}$
        }}
        {$f(z_b)$}
        {}

        \postlevel

        % Adversary attempts to guess from which distribution the value's image came from
        \send{}
        {$b'$}
        {\textsc{Output 1 iff} $b' = b$}

    \end{cryptogame}

    \begin{figure}
        \centering
        \sdinit{}
        \begin{tikzpicture}
            \sdbegin{}

            % Define symbols and names for the parties
            \newinst{A}{A}
            \newinst[2]{D}{D}
            \newinst[2]{C}{C$^\textsc{dist}$}
      
            % The distinguisher asks for the challenge
            \mess{C}{$z_b$}{D}
            \node[anchor=west] at (mess from) {\shortstack[l]{
                $z_0 \pickUAR X$ \\
                $z_1 \pickUAR Y$ \\
                $b \pickUAR \{0, 1\}$
            }};
      
            % The distinguisher applies f to z and relays the image to A
            \mess{D}{$f(z_b)$}{A}
            
            % Adversary distinguishes betweeen the two distributions
            \mess{A}{$b'$}{D}

            \mess{D}{$b'$}{C}
            \node[anchor=west] at (mess to) {\textsc{Output 1 iff} $b' = b$};
      
            \sdend{}
        \end{tikzpicture}
        \caption{Distinguisher reduction}
        \label{fig:distredux}
    \end{figure}


    In the game depicted by figure \ref{cryptogame:fdistin}, the adversary can make use of $f$, because it is \textsc{ppt}. Thus, if such a distinguisher exists, then it can be used to distinguish $X$ from $Y$, as shown in figure \ref{fig:distredux}. So, if $A$ could efficiently distinguish between $f(x)$ and $f(y)$, then $D$ can efficiently distinguish between $x$ and $y$. This contradicts the hypothesis $X \compindist Y$, which in turn means $f(X) \compindist f(Y)$.

\end{proof}

\pagebreak

\section{Pseudorandom Generators}

A deterministic function $G:\{0,1\}^{\lambda} \to \{0,1\}^{\lambda + l(\lambda)} $ is called a PseudoRandom Generator, or \prg{} iff:
\begin{itemize}
    \item $G$ runs in polinomial time
    \item $|G(s)| = \lambda + l(\lambda)$ % Well duh, we can see it by def...
    \item $G(U_{\lambda}) \approx_{c} U_{\lambda + l(\lambda)}$
\end{itemize}

\begin{cryptogame}{prg}{The pseudorandom game}{prg}
    
    \receive{\shortstack[l]{
        $x_0 \pickUAR G(\mathcal{U}_\lambda)$ \\
        $x_1 \pickUAR \mathcal{U}_{\lambda+l(\lambda)}$ \\
        $b \pickUAR \{0, 1\}$
    }}
    {$x_b$}
    {}

    \seqdelay

    \send{}
    {$b'$}
    {\textsc{Output 1 iff} $b = b'$}
    
\end{cryptogame}

So, if we take $s \pickUAR U_{\lambda}$, the output of $G$ will be indistinguishable from a random draw from $U_{\lambda}$.
