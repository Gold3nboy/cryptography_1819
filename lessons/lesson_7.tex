\mychapter{7}{Lesson 7}
\textbf{WARNING : I was absent} 
\begin{theorem}
    If $G$ is a PRG, then $F_{GGM}$ is a PRF.
\end{theorem}
(this definition should mean implicitly that the $k$ key has been chosen by the
challenger before the proof starts)
\begin{proof} TO BE REVIEWED.
    Now use the induction on the height $n$ of the GGM tree for $\nu(\lambda)
    \in \poly{\lambda} $ .\\
    
    \textbf{Base Case $\Rightarrow n=1$} follows by security of PRG $G$ function,
    because
    \[
        (F_{k}(0), F_{k}(1))=(G_{0}(k), G_{1}(k)) \approx_{c} U_{2\lambda}
    \]
    This means that, chosen $k$, the two values returned by $F$ are
    indistinguishable from 2 values taken at random and inserted in the
    truth table of a possible random function. Since these values are
    indistinguishables, the source functions are indistinguishables and $F_{k}$
    is pseudorandom.\\
    
    Now, for the \textbf{Inductive step} :\\
    \begin{lemma}
       Let $F':\{0,1\}^{n-1} \to \{0,1\}^{\lambda} $ be a PRF.
       Now define $F_{k}(x,y)=G_{x}(F_{k}'(y))$ where $F_{k}:\{0,1\}^{n} \to
       \{0,1\}^{\lambda} $.\\
       If $\{F_{k}'\}$ is a PRF so is $ \{F_{k}\}$.
    \end{lemma}
    
    Consider the following images:

    
    \newpage
    \begin{figure}[h!]
       \centering
       \sdinit{}
       \begin{tikzpicture}[scale=.6]
          % Define symbols and names for the parties
          \sdbegin{}
          \newinst{A}{$ \A $}
          \newinst[5]{B}{$C$} % Increase "5" to widenC
          
          \mess{A}{$x,y$}{B}
          \node[anchor=west] at (mess to) {  };
    
          \postlevel
          \mess{B}{z}{A}
          \node[anchor=west] at (mess from) {\shortstack[l]{
          		$ k \leftarrow\$ {0,1}^{\lambda}
          		   $ 
                \\
        $  z=F_{k}(x,y)=G_{x}(F_{k}'(y))  $ }};
          
          % Message from Bob to Alice, with computations by both sides
          
          \sdend{}
       \end{tikzpicture}
       \caption{$ \H\Y\B_{\F, \A}^{0}(\lambda) $}
       \label{fig:hib0}
    \end{figure}
    
    
    \begin{figure}[h!]
        \centering
       \sdinit{}
       \begin{tikzpicture}[scale=.6]
          % Define symbols and names for the parties
          \sdbegin{}
          \newinst{A}{$ \A $}
          \newinst[5]{B}{$C$} % Increase "5" to widen
          
          \mess{A}{$x,y$}{B}
          \node[anchor=west] at (mess to) {  };
    
          \postlevel
          \mess{B}{z}{A}
          \node[anchor=west] at (mess from) {\shortstack[l]{
                      $ R' \leftarrow\$ \R'(\lambda, n-1, \lambda) $ 
                \\
        $  z=H(x, y)=G_{x}(R'(y))  $ }};
          
          % Message from Bob to Alice, with computations by both sides
          
          \sdend{}
       \end{tikzpicture}
       \caption{$ \H\Y\B_{\R', G, \A}^{1}(\lambda) $}
       \label{fig:hib1}
    \end{figure}
    
    
    \begin{figure}[h!]
       \centering
       \sdinit{}
       \begin{tikzpicture}[scale=.5]
          % Define symbols and names for the parties
          \sdbegin{}
          \newinst{A}{$ \A $}
          \newinst[5]{B}{$C$} % Increase "5" to widen
          
          \mess{A}{$x,y$}{B}
          \node[anchor=west] at (mess to) {  };
    
          \postlevel
          \mess{B}{z}{A}
          \node[anchor=west] at (mess from) {\shortstack[l]{
                      $ R \leftarrow\$ \R(\lambda, n, \lambda) $ 
                \\
        $  z=R(x, y)  $ }};
          
          % Message from Bob to Alice, with computations by both sides
          
          \sdend{}
       \end{tikzpicture}
       \caption{$ \H\Y\B_{\R, \A}^{2}(\lambda) $}
       \label{fig:hib2}
    \end{figure}
    
    \begin{lemma}
$\H\Y\B^{0} \approx_{c} \H\Y\B^{1} \approx_{c} \H\Y\B^{2}$.
    \end{lemma}

    \begin{clm}
 $\H\Y\B^{0} \approx_{c} \H\Y\B^{1}$       
    \end{clm}

   Assume $ \exists .PPT. D_{01}$ that can distinguish $F_{k}$ and $H$ ; then
   there may exist a distinguisher $D$ as in the image which breaks the
   assumption made by inductive step.
\begin{figure}[h!]
   \centering
   \sdinit{}
   \begin{tikzpicture}
      % Define symbols and names for the parties
      \sdbegin{}
      \newinst{A}{$ D_{01}  $}
      \newinst[3]{B}{$ D $} % Increase "5" to widen
      \newinst[3]{C}{$\C  $}

      \mess{A}{$x, y$}{B}
      \node[anchor=west] at (mess to) {  };
      \mess{B}{$y$}{C}
      \node[anchor=west] at (mess to) {  };

      \postlevel
      \mess{C}{$z'$}{B}
      \node[anchor=west] at (mess from) {\shortstack[l]{
      		$  k \leftarrow\$ U_{\lambda}  $ 
            \\
            $z'=F_{k}(y)$ or $z'=R'(y)$
            \\
    $ R' \leftarrow\$ \R'(\lambda, n-1, \lambda)$;}};     
      \mess{B}{$G_{x}(z')$}{A}
      \node[anchor=west] at (mess to) {  };
      \sdend{}
   \end{tikzpicture}
\end{figure}

\begin{clm}
 $\H\Y\B^{1} \approx_{c} \H\Y\B^{2}$       
\end{clm}

For this proof we use the following simple lemma.
\begin{lem}
    If $ G:\{0,1\}^{\lambda} \to \{0,1\}^{2\lambda} $ is a PRG, then for any
    $t(\lambda) \in \poly{\lambda} $
    \[
        (G(k_{1}), ..., G(k_{t})) \approx_{c} (U_{2\lambda}, ..., U_{2\lambda})
    \]
    for $k_{1}, ..., k_{t} \leftarrow\$ U_{\lambda}$
\end{lem}

Assume that it exists a distinguisher $D_{1, 2}$ which is capable of distinguish
$H(x, y)$ and $R(x, y)$.\\

...TO REVIEW, NOT UNDERSTOOD AT ALL ...
\end{proof}

\section{CPA SECURITY}

Suppose $G$ is a PRG.\\
Given $Enc(k,m)=G(k) \xor m$, and a known $(\bar{m}, \bar{c})$ where
$c=G(k)\xor m$, then this function is not 2 time secure, since $c \xor
\bar{c}=m \xor \bar{m}$, easy to invert.\\
(??? some glue arguments missing ???)\\
\[
\]


\begin{figure}[h!]
   \centering
   \sdinit{}
   \begin{tikzpicture}[scale=0.7]
      % Define symbols and names for the parties
      \sdbegin{}
      \newinst{A}{$ \A $}
      \newinst[5]{B}{$ C^{cpa} $} % Increase "5" to widen
      
      \postlevel
      \mess{A}{$m_{i} \in \M$}{B}
      \node[anchor=west] at (mess to) {  };
      \postlevel
      \mess{B}{$c_{i}=Enc(k, m_{i})$}{A}
      \node[anchor=west] at (mess to) {  };
    \draw [->] (1.2,-3.3) to[out=240,in=110] (1.2,-1.7);
      
   \postlevel 
    \mess{A}{$m_{0}^{*}, m_{1}^{*} \in \M$}{B}
      \node[anchor=west] at (mess to) {  };

      \postlevel
      \mess{B}{$c*$}{A}
      \node[anchor=west] at (mess from) {\shortstack[l]{
      		$ k \leftarrow\$ \K   $ 
            \\
            $b \leftarrow\$ \{0,1\} $ 
            \\
    $ c=Enc(k, m_{b})   $ }};
      
      % Message from Bob to Alice, with computations by both sides
      \postlevel
      \mess{A}{$m_{i} \in \M$}{B}
      \node[anchor=west] at (mess to) {  };
      \postlevel
      \mess{B}{$c_{i}=Enc(k, m_{i})$}{A}
      \node[anchor=west] at (mess to) {  };
    \draw [->] (1.2,-8.3) to[out=240,in=110] (1.2,-6.3);
      \postlevel
      \mess{A}{$b' \in \{0,1\}$}{B}
      \node[anchor=west] at (mess to) {  };
      \sdend{}
   \end{tikzpicture}
   \caption{$Game_{\Pi, \A}^{cpa}(\lambda, b)$}
   \label{fig:cpa1}
\end{figure}
The adversary wins if finds which $m_{b}^{*}$ message was previously encrypted.
For sure $m_{i}$ can be equal to $m_{0}$ or $m_{1}$.\\
The two cyclic requests, before and after the starred messages, can be
arbitrarily long (also 0, there is no constraint).
\begin{definition}
  A scheme is CPA-secure if $Game_{\Pi, \A}^{cpa}(\lambda,0)
    \approx_{c}Game_{\Pi, \A}^{cpa}(\lambda,1)  $
\end{definition}

\begin{observation}
    No deterministic scheme can achieve CPA security.
\end{observation}
This is because the Adversary is not limited in what he can ask to the
Challenger: if he asks $m_{0}$ and $m_{1}$ before sending the starred messages,
he will know in advance the encrypted form of the messages, and he will be able
to distinguish with $ \P [Game^{cpa}]=1 $ the two games.\\

So, a way to obtain a CPA-secure encryption scheme consists of returning
different cyphertexts for the same message, and we can build a scheme which
tries to generate this kind of output using PRFs.\\

Consider the following SKE scheme $\Pi$.\\
Let $\F=\{ F_{k}:\{0,1\}^{n} \to \{0,1\}^{l} \}$ be a
PRF:
\begin{itemize}
    \item $k \leftarrow\$ U_{\lambda}$
    \item $Enc(k, m)$, picking a random $r \leftarrow\$ \{0,1\}^{n} $ with an
        output of $c=(c_{1}, c_{2})=(r, F_{k}(r) \xor m) \leftarrow\$
        \{0,1\}^{n+l}$
    
    \item $Dec(k, (c_{1}, c_{2}))=F_{k}(c_{1}) \xor c_{2}$
\end{itemize}
.
In this scheme, the only secret thing is $k$, which gives a \textit{flavour} to
the random function; Adversary can see $c=(r, F_{k}(r) \xor m)$, so he can see
$r$.
\begin{theorem}
    If $\F$ is a family of PRF functions, then $\Pi$ is  CPA-secure
\end{theorem}
To prove this, we have to prove that $Game_{\Pi, \A}^{cpa}(\lambda, 0)
\approx_{c} Game_{\Pi, \A}^{cpa}(\lambda, 1)$.
\begin{proof}
    Consider hybrid arguments
    \[
        \H\Y\B_{0} \equiv Game_{\Pi, \A}^{cpa}(\lambda, 0)
    \]
    and 
    \[ \H\Y\B_{1}\], like $\H\Y\B_{0}$ but with another distribution of $Enc(k,
    m_{b})$: \begin{itemize}
        \item picking $r \leftarrow\$ \{0,1\}^{n} $
        \item $R \leftarrow\$ \R(\lambda, n, l)$
        \item output obtained is $(r, R(r) \xor m_{b})$
    \end{itemize}
    and 
    \[ \H\Y\B_{2}(\lambda, b)\] which simply outputs $(r_{1}, r_{2})
    \leftarrow\$ U_{n+l}$.

\begin{lemma}
    $\H\Y\B_{0} \approx_{c} \H\Y\B_{1}$ for each $b \in
    \{0,1\} 
    $
\end{lemma}
\begin{proof}
    Suppose these two hybrids are distinguishable; but then we can use the related
    distinguisher to break the starting assumption saying $\F$ is a PRF.\\

    Since they are two similar games (but just for the encryption function), we can use
    the CPA game for this reduction.
\newpage
    Fix $b=0$ and build the following


\begin{figure}[h!]
   \centering
   \sdinit{}
   \begin{tikzpicture}[scale=0.65]
      % Define symbols and names for the parties
      \sdbegin{}
      \newinst{A}{$ \D_{01} $}
      \newinst[5]{B}{$ \D_{PRF} $} % Increase "5" to widen
      \newinst[4]{C}{$ \C $}

      \postlevel
      \mess{A}{$m_{i} \in \M$}{B}
      \node[anchor=west] at (mess to) {  };
      \postlevel
      \mess{B}{$r_{i}$}{C}
      \node[anchor=west] at (mess to) {  };

      \postlevel
      \mess{C}{$z_{i}$}{B}
      \node[anchor=west] at (mess from) {\shortstack[l] {
                  $z_{i}=F_{k}(r_{i})$
                  \\
                  $z_{i}=R(r_{i})$
          } };
      \postlevel:
      \mess{B}{$c_{i}=(c_{i_{1}}, c_{i_{2}})=(z_{i}, z_{i} \xor m_{i})$}{A}
      \node[anchor=west] at (mess to) {  };
    \draw [->] (1.2,-5.7) to[out=240,in=110] (1.2,-1.6);
    \postlevel
    \mess{A}{$m_{0}^{*}, m_{1}^{*}$}{B}
      \node[anchor=west] at (mess to) {  };

    \postlevel
    \mess{B}{$r^{*}$}{C}
      \node[anchor=west] at (mess to) {  };
    
    \postlevel
    \mess{C}{$z^{*}$}{B}
      \node[anchor=west] at (mess to) {  };
      \postlevel
      \mess{B}{$c^{*}=(z^{*}, z^{*} \xor m_{0})$}{A}
      \node[anchor=west] at (mess from) {\shortstack[l]{
      		$    $ 
            \\
            $    $ }};
      
%      \postlevel
%      \mess{A}{$b' \in {0,1}$}{B}
%      \node[anchor=west] at (mess to) {  };
      \sdend{}
      \sdend{}
   \end{tikzpicture}
  % \caption{}
  % \label{fig:}
\end{figure}

After the last message, $D_{01}$ will reply with a $b' \in \{0,1\}$ which
expresses which one of the encryption functions has been used; since this result
can be used for solving the PRF game, the starting assumption fails and the
lemma is proven.

\end{proof}

\begin{lemma}

    $\H\Y\B_{1} \approx_{c} \H\Y\B_{2}$
    
\end{lemma}
\begin{proof}
Even if the output of the first  hybrid is $(r_{i}, R(r_{i}) \xor m_{b})$,
the distribution of the second member of this couple doesn't depend on $m_{b}$,
hence we can assume that $R(r_{i}) \xor m_{b} \equiv R(r_{i})$.\\

The difference between $\H\Y\B_{1}$ and $\H\Y\B_{2}$ comes if we play the game
once more and we pick up the same $c_{1}$ of the first game: while in
$\H\Y\B_{2}$ $r_{2}$ is independent from $r_{1}$ and will be different from the
previous $r'_{2}$  with probability near to 1, in $\H\Y\B_{1}$ ( if $r_{i}$ is
the same of the previous game) 
$R(r_{i})$ will be the same of the previous game (because $R()$ is a
random function).\\

Anyway, we can show that this "collision" happens with very low probability.\\

Call \textbf{REPEAT}  this event of collision of $r_{i}$ between 2 consecutive
games.\\
To show the statement of the lemma, it suffices to show that 
$P[REPEAT] \in \negl{\lambda} $, therefore the two distributions are
indistinguishable (or distinguishable with a negligible probability) .\\

In fact, if I make $q$ queries to $\H\Y\B_{1}$, 
\begin{gather*}
P[REPEAT]=P[ \exists i, j \in q \text{ such that } r_{i}=r_{j}] \leq \\
\leq \sum_{i,j \wedge i \not= j}^{} \P [ r_{i}=r_{j} ]   = Col(U_{n}) = \\
= \sum_{i \wedge j, i \not= j} \sum_{e \in \{0,1\}^{n}} \P [ r_{1}=r_{2}=e ]  = \\
=\sum_{i \wedge j, i \not= j} \sum_{e \in \{0,1\}^{n}} \P[ r=e ]^{2}  = \\
= {q \choose 2} 2^{n} \frac{1}{2^{2n}} = \\
={q\choose 2} 2^{-n} \leq \\
\leq q^{2}2^{-n} \in \negl{\lambda} 
\end{gather*} 
\end{proof}

Since $\H\Y\B_{0} \equiv Game(\lambda, 0)$ but the same proofs can be made for
$Game(\lambda, 1)$,  
$ Game(\lambda, 0) \equiv \H\Y\B_{0}(Game(\lambda, 0)) \approx_{c} \H\Y\B_{2} \approx_{c} 
\H\Y\B'_{0}(Game(\lambda,1)) \equiv Game(\lambda, 1)$
\end{proof}

