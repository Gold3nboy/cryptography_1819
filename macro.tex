\usepackage{titletoc}
\usepackage{hyperref}
\usepackage{amsmath}
\usepackage{amsfonts}
\usepackage{amsthm}
\usepackage{amssymb}
\usepackage{thmtools}
\usepackage{acronym}
\usepackage{multirow}
\usepackage{cleveref}
\usepackage{graphicx}
\usepackage{mathtools}
\usepackage{cancel}
\usepackage{ifthen}
\usepackage{etoolbox}
\usepackage{xcolor}
\usepackage{subcaption}
\usepackage{calc}
\usepackage{tikz}
\usepackage{preview}
\usepackage{color}
\usepackage{framed}
\hypersetup{colorlinks=false}
\usepackage{preview}
\usepackage{stackengine}
\usepackage{scalerel}
\usetikzlibrary{tikzmark,decorations.pathreplacing}

\makeatletter
\usepackage{framed}
\usepackage{color}
\newcommand{\review}[1]{
  \definecolor{shadecolor}{rgb}{1,1,0} % this is yellow
  \begin{shaded}
  REVIEW: #1
 \end{shaded}
}

\titlecontents{chapter}[0em]{\lsstyle\smallskip\bfseries}%\vspace{1cm}%
      {\contentslabel[\chaptername~\thecontentslabel]{0em}\hspace{5.6em}}%
      {\renewcommand\thecontentslabel{\relax}\itshape}%numberless%
      {\hfill\contentspage}[\medskip]%

      
% Cryptography arrow symbol denoting a uniformly random sampling action
\newcommand\pickUAR{\mathrel{{\leftarrow}\vcenter{\hbox{\tiny\rmfamily\upshape\!\!\$}}}}


%this is used to create space between \forall, \exist [ and others ] with the
      %next character
\let\eps\varepsilon
\newcommand\+{\mkern2mu}

\newcommand{\ie}{\textit{i.e.}, }
\newcommand{\eg}{\textit{e.g.}, }

\newcommand{\divides}{\big|}

\newcommand{\biglor}{\bigvee}
\newcommand{\bigxor}{\bigoplus}

\newcommand{\Definition}{\ensuremath{\mathop{=}\limits^{\scriptscriptstyle \triangle}}}

\renewcommand{\vec}[1]{\overline{#1}}
\newcommand{\norm}[1]{\left\lVert{#1}\right\rVert}

\renewcommand{\epsilon}{\varepsilon}

\newcommand{\comma}{\ensuremath{, \allowbreak}}
\newcommand{\from}{\leftarrow}
\newcommand{\toandfrom}{\rightleftarrows}

\newcommand{\A}{\mathcal{A}}
\newcommand{\B}{\mathcal{B}}
\newcommand{\C}{\mathcal{C}}
\newcommand{\D}{\mathcal{D}}
\newcommand{\F}{\mathcal{F}}
\newcommand{\G}{\mathcal{G}}
\renewcommand{\H}{\mathcal{H}}
\newcommand{\K}{\mathcal{K}}
\newcommand{\M}{\mathcal{M}}
\renewcommand{\P}{\mathcal{P}}
\newcommand{\R}{\mathcal{R}}
\newcommand{\U}{\mathcal{U}}
\newcommand{\V}{\mathcal{V}}
\newcommand{\X}{\mathcal{X}}
\newcommand{\Y}{\mathcal{Y}}
\newcommand{\Z}{\mathcal{Z}}

\newcommand{\RO}{\mathrm{RO}}

\newcommand{\Naturals}{\mathbb{N}^{+}}
\newcommand{\NaturalsZ}{\mathbb{N}}
\newcommand{\Integers}{\mathbb{Z}}
\newcommand{\Reals}{\mathbb{R}}
\newcommand{\Bool}{\{0,1\}}
\newcommand{\QR}[1][p]{\mathbb{QR}_{#1}}

\newcommand{\IntegersPrimeGroup}[1][p]{\Integers^{\star}_{#1}}

\newcommand{\Coll}{\mathrm{Coll}}
\newcommand{\Ext}{\mathrm{Ext}}
\newcommand{\Gen}{\mathrm{Gen}}
\newcommand{\KGen}{\mathrm{KGen}}
\newcommand{\KeyGen}{\mathrm{KeyGen}}
\newcommand{\Mac}{\mathrm{Mac}}
\newcommand{\Sign}{\mathrm{Sign}}
\newcommand{\Vrfy}{\mathrm{Vrfy}}
\newcommand{\Enc}{\mathrm{Enc}}
\newcommand{\Dec}{\mathrm{Dec}}
\newcommand{\Trans}{\mathrm{Trans}}

\renewcommand{\Game}{\G}

\newcommand{\IDSGame}{\Game^{\mathrm{id}}_{\Pi, \Adv}}

\newcommand{\GenericEncScheme}{\Pi}
\newcommand{\GenericKeyGen}{\Gen}
\newcommand{\GenericEnc}{\Enc}
\newcommand{\GenericDec}{\Dec}
\newcommand{\GenericEncSchemeTuple}{\GenericEncScheme = \left( \GenericKeyGen\comma \GenericEnc\comma \GenericDec \right)}

\newcommand{\GenericMACScheme}[1][]{\Pi_{#1}}
\newcommand{\GenericMac}{\Mac}
\newcommand{\GenericVrfy}{\Vrfy}
\newcommand{\GenericAuthSpace}{\Phi}
\newcommand{\GenericMACSchemeTuple}[1][]{\GenericMACScheme[#1] = \left( \GenericKeyGen\comma \GenericMac\comma \GenericVrfy \right)}

\newcommand{\SKEScheme}[1][]{\Pi_{#1}}
\newcommand{\SKEKeyGen}{\Gen}
\newcommand{\SKEEnc}{\Enc}
\newcommand{\SKEDec}{\Dec}
\newcommand{\SKESchemeTuple}[1][]{\SKEScheme[#1] = \left( \SKEKeyGen \comma \SKEEnc \comma \SKEDec \right)}

\newcommand{\SKEGameOneTime}{\Game^{\scriptstyle \text{one time}}_{\SKEScheme, \Adv}}

\newcommand{\CCAScheme}{\Pi'}
\newcommand{\CCAKeyGen}{\Gen'}
\newcommand{\CCAEnc}{\Enc'}
\newcommand{\CCADec}{\Dec'}
\newcommand{\CCASchemeTuple}{\CCAScheme = \left( \CCAKeyGen \comma \CCAEnc \comma \CCADec \right)}

\newcommand{\PRFGameReal}{\Game^{\scriptstyle \text{real}}_{\F, \Adv}}
\newcommand{\PRFGameRand}{\Game^{\scriptstyle \text{rand}}_{\F, \Adv}}

\newcommand{\PKEScheme}{\Pi}
\newcommand{\PKEKeyGen}{\Gen}
\newcommand{\PKEEnc}{\Enc}
\newcommand{\PKEDec}{\Dec}
\newcommand{\PKESchemeTuple}{\PKEScheme = \left( \PKEKeyGen \comma \PKEEnc \comma \PKEDec \right)}

\newcommand{\PKEGameCPA}{\Game_{\Adv, \Pi}^{\text{CPA \\ one-time}}}
\newcommand{\PKEGameCCA}[1]{\Game_{\Adv, \Pi}^{\text{CCA{#1}}}}

\newcommand{\SignScheme}{\Pi}
\newcommand{\SignSchemeKGen}{\KGen}
\newcommand{\SignSchemeSign}{\Sign}
\newcommand{\SignSchemeVrfy}{\Vrfy}
\newcommand{\SignSchemeTuple}{\SignScheme = \left( \SignSchemeKGen \comma \SignSchemeSign \comma \SignSchemeVrfy \right)}

\newcommand{\RSA}{\mathrm{RSA}}
\newcommand{\RSAGen}{\Gen_{\RSA}}
\newcommand{\RSAfun}{f_{\RSA}}
\newcommand{\RSAinv}{f_{\RSA}^{-1}}
\newcommand{\RSAtuple}{(\RSAGen, \RSAfun, \RSAinv)}

\newcommand{\CPAGame}{\Game^{\scriptstyle \text{cpa}}_{\SKEScheme,\Adv}}
\newcommand{\CCAGame}{\Game^{\scriptstyle \text{cca}}_{\SKEScheme,\Adv}}

\newcommand{\CFPTuple}{\Pi = (\Gen, f_0, f_1)}
\newcommand{\CFPGame}{\Game^{\scriptstyle \text{CF}}_{\Pi, \Adv}}

\newcommand{\CRHGame}{\Game^{\scriptstyle \text{CR}}_{\H,\Adv}}
\newcommand{\CRDMGame}{\Game^{\scriptstyle \text{CR}}_{{\scriptstyle \text{DM}}, \Adv}}

\newcommand{\UFCMAGame}{\Game^{\scriptstyle \text{ufcma}}_{\GenericMACScheme,\Adv}}

\newcommand{\UHFGameReal}{\Game^{\scriptstyle \text{real}}_{\F(\H),\Adv}}
\newcommand{\UHFGameHybrid}{H_{\$,\H,\Adv}}
\newcommand{\UHFGameRand}{\Game^{\scriptstyle \text{rand}}_{\$,\Adv}}

\newcommand{\OTPEnc}{\Enc}
\newcommand{\OTPDec}{\Dec}

\newcommand{\Feistel}[1][F]{\psi_{#1}}

\newcommand{\Group}{\mathbb{G}}
\newcommand{\Field}{\mathbb{F}}
\newcommand{\GroupTuple}{(\Group \comma g \comma q)}
\newcommand{\GroupGen}{\GroupTuple \from \mathrm{GroupGen}(1^{\lambda})}

\newcommand{\Bilin}{\mathrm{Bilin}}
\newcommand{\BilinearGroup}{\Group}
\newcommand{\BilinearGroupBis}{\BilinearGroup_{T}}
\newcommand{\BilinearMap}{\hat{e}}
\newcommand{\BilinearGroupTuple}{\left( \BilinearGroup, \BilinearGroupBis, q, g, \BilinearMap \right)}

\newcommand{\WatersKGen}{\mathrm{BiGen}}

\newcommand{\pk}{pk}
\newcommand{\sk}{sk}

\newcommand{\params}{params}

\newcommand{\pr}{\mathop{\mathrm{Pr}}}
%\renewcommand{\Pr}[2][]{\ensuremath{\pr\limits_{#1} \left[ {#2} \right]}}

\newcommand{\rand}[1]{\from {\scriptstyle \$} {#1}}
\newcommand{\abs}[1]{\left|{#1}\right|}
\newcommand{\poly}{\mathrm{poly}}

\newcommand{\DotProduct}[2]{\left< {#1}, {#2} \right>}
\newcommand{\Generator}[1]{\left< {#1} \right>}

\newcommand{\repr}[1]{\left< {#1} \right>}

\newcommand{\CompInd}{\approx_c}
\newcommand{\StatInd}{\approx_s}

\newcommand{\Adv}{\A}
\newcommand{\Distinguisher}{\D}

\newcommand{\negl}[1]{\mathrm{negl}\left({#1}\right)}

\newcommand{\xor}{\oplus}

%https://tex.stackexchange.com/questions/137073/writing-mod-in-congruence-problems-without-leading-space
\newcommand{\Mod}[1]{\ (\mathrm{mod}\ #1)}

%========================================================================================================



%\renewcommand\thmt@mklistcmd{%
%  \@xa\protected@edef\csname l@\thmt@envname\endcsname{% CHECK: why p@edef?
%    \@nx\@dottedtocline{1}{1.5em}{\@nx\thmt@listnumwidth}{\thmt@thmname}{mu}%
%  }%
%  \ifthmt@isstarred
%    \@xa\def\csname ll@\thmt@envname\endcsname{%
%      \protect\numberline{\thmt@thmname\protect\let\protect\autodot\protect:}%
%      \ifx\@empty\thmt@shortoptarg\else\protect\thmtformatoptarg{\thmt@shortoptarg}\fi
%    }%
%  \else
%    \@xa\def\csname ll@\thmt@envname\endcsname{%
%      % \thmt@thmname\ \csname the\thmt@envname\endcsname: \hfil%
%      \ifx\@empty\thmt@shortoptarg\else\thmt@shortoptarg\fi
%    }%
%  \fi
%  \@xa\gdef\csname thmt@contentsline@\thmt@envname\endcsname{%
%    \thmt@contentslineShow% default:show
%  }%
%}
\makeatother


\theoremstyle{plain}
\declaretheorem[qed=\ensuremath{\diamond}]{theorem}
\declaretheorem[qed=\ensuremath{\diamond}]{lemma}
\declaretheorem[qed=\ensuremath{\diamond}]{corollary}
\declaretheorem[qed=\ensuremath{\diamond}]{fact}
\declaretheorem[qed=\ensuremath{\diamond}]{claim}
\declaretheorem[qed=\ensuremath{\diamond}]{observation}
\declaretheorem[qed=\ensuremath{\diamond}]{proposition}

\crefname{thm}{theorem}{theorems}
\Crefname{thm}{Theorem}{Theorems}

\crefname{lem}{lemma}{lemmas}
\Crefname{lem}{Lemma}{Lemmas}

\crefname{cor}{corollary}{corollarys}
\Crefname{cor}{Corollary}{Corollarys}

\crefname{fct}{fact}{facts}
\Crefname{fct}{Fact}{Facts}

\crefname{clm}{claim}{claims}
\Crefname{clm}{Claim}{Claims}

\crefname{obs}{observation}{observations}
\Crefname{obs}{Observation}{Observations}

\crefname{prop}{proposition}{propositions}
\Crefname{prop}{Proposition}{Propositions}


\theoremstyle{definition}
\declaretheorem[qed=\ensuremath{\diamond}]{definition}
\declaretheorem[qed=\ensuremath{\diamond}]{construction}

\crefname{defn}{definition}{definitions}
\Crefname{defn}{Definition}{Definitions}

\crefname{cons}{construction}{constructions}
\Crefname{cons}{Construction}{Constructions}


% \theoremstyle{plain}
\newtheorem{thm}{Theorem}
\newtheorem{lem}{Lemma}
\newtheorem{cor}{Corollary}
\newtheorem{fct}{Fact}
\newtheorem{clm}{Claim}
\newtheorem{obs}{Observation}
\newtheorem{prop}{Proposition}
\newtheorem{question}{Question}
\newtheorem{exercise}[theorem]{Exercise}
\newtheorem{solution}{Solution}
\newtheorem{example}{Example}
% \theoremstyle{definition}
\newtheorem{defn}{Definition}
\newtheorem{cons}{Construction}


%-------------------DEBUG COMMANDS----------------------------
\newcounter{t0d0_counter}
\newcommand{\notodo}[1]{
}
\newcommand{\todo}[1]{
  \stepcounter{t0d0_counter}
  \definecolor{shadecolor}{rgb}{1,1,0} % this is yellow
  \begin{shaded}
  T0D0 \arabic{t0d0_counter}: #1
  \end{shaded}
}

%----------------------IMAGES-----------------------------------
\usepackage[underline=false]{pgf-umlsd}
\usepackage{tikz}

% From http://tex.stackexchange.com/questions/164707/how-to-use-greek-letters-in-pgf-umlsd-or-generally-terms-starting-with
\renewcommand{\mess}[4][0]{
  \stepcounter{seqlevel}
  \path
  (#2)+(0,-\theseqlevel*\unitfactor-0.7*\unitfactor) node (mess from) {};
  \addtocounter{seqlevel}{#1}
  \path
  (#4)+(0,-\theseqlevel*\unitfactor-0.7*\unitfactor) node (mess to) {};
  \draw[->,>=angle 60] (mess from) -- (mess to) node[midway, above]
  {#3};
  \node (\detokenize{#3} from) at (mess from) {};
  \node (\detokenize{#3} to) at (mess to) {};
}
% From http://tex.stackexchange.com/questions/98525/pgf-umlsd-and-externalize
\newcommand{\sdinit}{%
   \pgfdeclarelayer{umlsd@background}%
   \pgfdeclarelayer{umlsd@threadlayer}%
   \pgfsetlayers{umlsd@background,umlsd@threadlayer,main}%
}
\newcommand{\sdbegin}{%
   \setlength{\unitlength}{1cm}%
   \tikzstyle{sequence}=[coordinate]%
   \tikzstyle{inststyle}=[rectangle, draw, anchor=west, minimum
   height=0.8cm, minimum width=1.6cm, fill=white, 
   drop shadow={opacity=1,fill=black}]%
   \ifpgfumlsdroundedcorners%
      \tikzstyle{inststyle}+=[rounded corners=3mm]%
   \fi%
   \tikzstyle{blockstyle}=[anchor=north west]%
   \tikzstyle{blockcommentstyle}=[anchor=north west, font=\small]%
   \tikzstyle{dot}=[inner sep=0pt,fill=black,circle,minimum size=0.2pt]%
   \global\def\unitfactor{0.6}%
   \global\def\threadbias{center}%
   % reset counters
   \setcounter{preinst}{0}%
   \setcounter{instnum}{0}%
   \setcounter{threadnum}{0}%
   \setcounter{seqlevel}{0}%
   \setcounter{callevel}{0}%
   \setcounter{callselflevel}{0}%
   \setcounter{blocklevel}{0}%
   % origin
   \node[coordinate] (inst0) {};%
}
\newcommand{\sdend}{%
   \begin{pgfonlayer}{umlsd@background}%
      \ifnum\value{instnum}>0%
         \foreach \t [evaluate=\t] in {1,...,\theinstnum}{%
            \draw[dotted] (inst\t) -- ++(0,-\theseqlevel*\unitfactor-2.2*\unitfactor);%
         }%
      \fi%
      \ifnum\value{threadnum}>0%
         \foreach \t [evaluate=\t] in {1,...,\thethreadnum}{%
            \path (thread\t)+(0,-\theseqlevel*\unitfactor-0.1*\unitfactor) node (threadend) {};%
            \tikzstyle{threadstyle}+=[threadcolor\t]%
            \drawthread{thread\t}{threadend}%
         }%
      \fi%
   \end{pgfonlayer}%
}



\newcounter{listcount} \newcounter{totcount}
\newcommand{\printarray}[2][1em]{% \printarray[<width>]{<array list>}
  \unskip \setcounter{totcount}{0}% Reset totcount counter
  \renewcommand*{\do}[1]{\stepcounter{totcount}}% Count elements
  \docsvlist{#2}% Process list a first time to obtain # of elements
  \setcounter{listcount}{0}% Reset listcount counter
  \renewcommand*{\do}[1]{%
    \stepcounter{listcount}% Move to next element
    \framebox[#1][c]{\rule{0pt}{1.5ex}\smash{\ensuremath{##1}}}%
    \ifnum\value{listcount}<\value{totcount}\thickspace\fi
  }
  \docsvlist{#2}% Process list a second time to typeset each element
}


% \newcommand{\thmsymbol}{\( \diamond \)}
% \newenvironment{definition}{\begin{defn}%
% \renewcommand{\qedsymbol}{\thmsymbol}\pushQED{\qed}}%
% {\popQED\end{defn}}
% \newenvironment{construction}{\begin{cons}%
% \renewcommand{\qedsymbol}{\thmsymbol}\pushQED{\qed}}%
% {\popQED\end{cons}}
% \newenvironment{theorem}{\begin{thm}%
% \renewcommand{\qedsymbol}{\thmsymbol}\pushQED{\qed}}%
% {\popQED\end{thm}}
% \newenvironment{lemma}{\begin{lem}%
% \renewcommand{\qedsymbol}{\thmsymbol}\pushQED{\qed}}%
% {\popQED\end{lem}}
% \newenvironment{corollary}{\begin{cor}%
% \renewcommand{\qedsymbol}{\thmsymbol}\pushQED{\qed}}%
% {\popQED\end{cor}}
% \newenvironment{fact}{\begin{fct}%
% \renewcommand{\qedsymbol}{\thmsymbol}\pushQED{\qed}}%
% {\popQED\end{fct}}
% \newenvironment{claim}{\begin{clm}%
% \renewcommand{\qedsymbol}{\thmsymbol}\pushQED{\qed}}%
% {\popQED\end{clm}}
% \newenvironment{observation}{\begin{obs}%
% \renewcommand{\qedsymbol}{\thmsymbol}\pushQED{\qed}}%
% {\popQED\end{obs}}
% \newenvironment{proposition}{\begin{prop}%
% \renewcommand{\qedsymbol}{\thmsymbol}\pushQED{\qed}}%
% {\popQED\end{prop}}


% \renewcommand{\thmtformatoptarg}[1]{ #1}
\setcounter{chapter}{4}
